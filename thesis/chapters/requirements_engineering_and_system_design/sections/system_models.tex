\section{System Models}

After eliciting and specifying requirements, this section presents scenarios and analysis models representing how the system satisfies these requirements.
These models aim to improve the comprehension of FLOps on the application level instead of the solution one.
The models do not depict concrete implementation.
Instead, they show a simplified conceptual representation of FLOps' architecture and workflows.
This includes involved components and their relationships.
These are not identical to the actual implementation.
The goal is to improve comprehension of the system instead of showing overwhelmingly verbose intricate details that might change in future updates.

System models aim to build what is called an analysis model.
The analysis model has three distinct parts
The scenarios and use cases form the functional model.
Class and object models make the analysis object model.
State machines and sequence diagrams create the dynamic model of the system. \cite{book:bruegge}

\subsection{Scenarios}

% TODO paraphraze!
The goal of scenarios is to enhance the understanding of the proposed system by looking at a concrete set of common use cases [BD09].
These scenarios are also written in natural language to further increase the comprehension of the planed end user experience while working with the new system.
Scenarios can also be used to build models. We differentiate between two kinds of scenarios.
Demo scenarios that illustrate the achieved use cases of the new system.
Visionary scenarios portrait an almost utopian experience that will not be realized by this thesis, but lay the foundation for future work.


% scenario Name.
% participating actor instances:

% Flow of events:
\subsubsection{Demo Scenario A: FLOps workflow for a FL newcomer}

TODO rephrase this so it is maybe more university than highschool? - SC prof that is not specifialized in ML or FL

John is a computer science teacher at a highschool.
He is new to FL, he has some experience in ML but no in MLOps, DevOps or automation.
During the corona pandemic he had the change to try out different digital solutions and approaches to motivate and teach his students.
He wants to figure out if there are patterns and correlations between specific approaches and the number and quality of completed homework.
Because doing this analysis by hand is too tedious and time consuming he wants to use ML.
Specific local device metrics such as how long students have programs open that are relevant for homework and at what time students make the biggest progress
or how long each exercise took for students to complete are of his interest.
He creates a ML code repository where he defines the model architecture and how it should be trained.
He verifies if his model training approach works by providind mock data.

He is well aware that he is neither legaly nor morally allowed to access this private student data.
However, these patterns and correlations might make a significant difference for his students success.
He read online that FL can overcome such issues.
He wants to set this up by instructing his students to allow their devices to gather these metrics and to train local models based on this data.
Therefore, John can get his global model and find patterns without breaking the privacy and trust of his students.

The students he works with have diverse setups for online homework.
Some have might gaming PCs, other old or borrowed laptops, some savy kids use very unique devices.
So he needs to to enable FL on different devices in a easy and non breaking way for his students to replicate at home or during CS class.

He want to use established solutions instead of creating a custom solution from scratch.
He notices that there are several FL frameworks but no easy platform to do FL, until he discovers FLOps.

He installs docker on his students devices, starts a local python script to scrape the metrics he is interested in.
Now student machines collect the correct data locally.
He setsup a oakestra cluster at school and shows his students how to turn their devices into worker nodes.

After a week of collecting data he starts up his FLOps management components with a single command via the CLI.
He sends a SLA request to his FLOps manager via the CLI.
He assumes that only halve of his students have working connection to the orchestrator.
He configures his SLA to have a minimum of 14 learners and a maximum of 28 learners.
He wants to do a test run so he keeps everything else as is.
Thus he will create a FLOps project that will perform classic FL for 3 FL rounds.
Each round will trigger his ML model that will train 10 rounds locally.
It total each learner will train for 30 rounds.
As part of this SLA he provides a link to his github repository where he stored his ML code.
He needed to adjust his code slightly to satisfy FLOps' requirements.

He sends of his requests and observes via the project observer that a new project has been started.
He goes to the tracking server url shows in the observer.
There he sees a modern GUI showing the first round of training.
This first round uses 21 learners.

After 30 minutes the FL training is completed and he can see in the GUI that his model has now a 65\% accuracy.
John wants to have a higher accuracy.
He starts another FLOps project that now performs 6 rounds instead of 3.
He notices that the second time the training starts sooner because the relevant components have already been build and reused from the integrated image registry.
After the training has completed john sees that the new model reached 87\% accuracy.
He wants to test it out straight away.
This time he added the optional post training steps in the SLA to automatically wrap the trained model as an inference server and deploy it in the orchestrator.
He finds this inference server and sends a request to it to check if things work out as planned.
The inference server returns his requests with the correct prediction and presents the first interesting pattern to him.
Now John can continue finding new patterns and correlations to help his students to learn and him to teach better while preserving privacy.




\subsubsection{Demo Scenario B}

\subsubsection{Visionary Scenario}
TODO
% Maybe split scenarios up by target group - enduser no experience - FL researcher - FL dev

\subsection{Use Case Model}

UML use case diagrams visualize the use of all significant available functions of a system from the user's perspective.
They are based on the functional requirements.
This means that every function, whether directly triggered by the user or an internal system function, that leads to observable changes and results for the user is depicted.
Use case diagrams aim to showcase available and consequential functionality as seen by the user in a compact representation. \cite{book:bruegge}

\begin{figure}[p]
    \begin{adjustwidth}{-0.1\paperwidth}{-0.1\paperwidth}
        \centering
        \includegraphics[width=0.9\paperwidth]{uml_use_case_diagram.png}
        \caption{FLOps UML Use Case Diagram}
        \label{fig:uml_use_case_diagram}
    \end{adjustwidth}
\end{figure}

Figure \ref{fig:uml_use_case_diagram} shows the Use Case diagram for FLOps.
The white use cases represent the functionalities the external users can directly trigger.
The grey use cases are internal system actions that are directly visible to users or lead to visible results.
They get triggered as a result of user actions.
For example, the user knows that FLOps is performing FL training by inspecting different provided outlets, such as the GUI.
FLOps tracks the training progress and results.
These logged artifacts become incrementally visible to the user who inspects the GUI.
Thus, the user knows that FLOps is currently performing FL training and logging.
Use cases inside the GUI boundary are directly accessible via the GUI. 
The same applies to the API boundary.
Other tasks are executed and accessible via FLOps combined with its orchestrator.
Use cases that involve developing or modifying FLOps itself are not explicitly portrayed.
The depicted User actor represents end users of varying FL expertise.
This actor includes FL developers and researchers.
The core use case is starting an FL project.
This activity starts a chain of events, such as building an FL-enabled container image, creating and deploying the learners and aggregator(s), and performing the FL training.
During training, FLOps tracks the model and system metrics, which the user can monitor and evaluate in the GUI.
After training, the model can be containerized and deployed as an inference server.
The user can access this trained model and request services from its inference server.


\subsection{FLOps Overview}

Thanks to the gathered requirements, scenarios, and use cases the reasons and needs for FLOps are clear.
Depicting and covering all aspects at once would require a steep learning curve and complicated images and explanations.
Therefore we will avoid large singular models and explanations but describe and depict FLOps piece by piece, from coarse to fine grained.
To avoid confusion by focusing on individual aspects before understanding the raw outline of the entire system,
this subsection shows a heavily abstracted overview of the big picture.
These simplified concepts are discussed in detail in their own following sections throughout this thesis.

\begin{figure}[h]
    \centering
    \includegraphics[width=1.0\textwidth]{primer_flops.png}
    \caption{FLOps Coarse Structural Overview}
    \label{fig:flops_structure_overview}
\end{figure}

Figure \ref{fig:flops_structure_overview}
The FLOps system is realized via the interactions and relationships between the FLOps Management, the orchestrator, and the worker nodes.
The orchestrator is Oakestra.
The FLOps management is a composition of different components (containers).
Its goals and responsibilities are to manage and store FLOps processes.
The management components coordinate different automatic processes and events and store build container images and training results such as metrics and trained models.
These managerial components do not perform the FL training.
They delegate and distribute computation to orchestrated worker nodes.
The FLOps manager uses the orchestrator to create, (un)deploy, and remove different components.
Especially the computationally heavy image builds and FL training is spread across the worker nodes.
The GUI and inference servers are also run on worker nodes.

\begin{figure}[h]
    \centering
    \includegraphics[width=0.8\textwidth]{simple_builder.png}
    \caption{Simplified FLOps Image Builder Processes}
    \label{fig:flops_simple_image_builder}
\end{figure}

Figure \ref{fig:flops_simple_image_builder} shows a simplified overview of FLOps' image builder processes.
The container images get build on worker nodes.
The build process occurs inside of a container, thus special requirements arise.
Remember that the user only provides ML code not FL code.
FLOps' image builder clones the user ML code, augments it to support FL, handles certain dependency issues, and builds multi platform container images.
This builder is able to build FL actors, i.e. the aggregator and learner images as well as an inference server for the trained model.
These images get pushed to the FLOps image registry.
When the learners, aggregators, or inference servers are needed their corresponding images get pulled from that registry onto a orchestrated worker node and executed.

\begin{figure}[h]
    \centering
    \includegraphics[width=0.8\textwidth]{simple_data_management.png}
    \caption{Simplified FLOps Local Data Management}
    \label{fig:flops_simple_data_management}
\end{figure}

Figure \ref{fig:flops_simple_data_management} shows a simplified overview how local training data is managed.
FLOps is targeted for practical real FL application.
Thus it does not expect users to provide data as part of their ML repositories.
Instead, users need to coordinate with real data providers on the orchestrated worker nodes.
The figure shows that a learner container is deployed on a worker node.
The learner container itself has no data. 
FLOps cooperates with the orchestrator and deploys a ML Data Server before training on user specified worker nodes.
This data server is reachable by nearby devices via an API.
Devices can send their data to this data server.
The data server will store this data on the local machine.
During FL training FLOps, the augmented learner container will fetch the local data via the data server.
This local data will be forwarded to the user ML code for preprocessing and training.



\subsection{Analysis Object Models}
% TODO Paraphraze!
To increase the understanding of the underlying structure of the proposed
system UML class diagrams are used to visualize the main components and
their relationships [BD09].
The analysis model represents the system under development from the user’s point of view. The
"analysis object model is a part of the analysis model and focuses on the individual concepts that
are manipulated by the system, their properties and their relationships. The analysis object
model, depicted with UML class diagrams, includes classes, attributes, and operations. The
analysis object model is a visual dictionary of the main concepts visible to the user."

FLOps covers various different aspects that need to be understood individually to be able to comprehend their need for the whole picture.



\begin{figure}[h]
    \centering
    \includegraphics[width=1.0\textwidth]{uml_analysis_object_model_core.png}
    \caption{FLOps Core UML Analysis Object Model}
    \label{fig:uml_core_analysis_object_model}
\end{figure}

Figure \ref{fig:uml_core_analysis_object_model} shows the core FLOps UML Analysis Object Model.
Following models will explain more concrete details about individual core components. 
The main workflow is represented and grouped via a FLOps Project.
Such a project links all necessary FL and ML/DevOps components together to power one entire FL user request.
A project contains information about the user who requested it, the target platforms that should be supported (e.g. ARM/AMD) or what steps FLOps should perform after training.
If no steps are specified the FLOps project counts as completed after training.
Available post-training steps include building a containerized image for the trained model 
and deploying an inference server to serve the trained model.
The ML Model Flavor is an indicator to tell FLOps what ML framework to expect and to work with.
Examples include Keras, Sklearn, or Pytorch.
Each project is associated with exactly one ML Code Repository.
This repository can be owned by the user or be a public one.
Thus, multiple users can reuse the same repository and each user can create multiple FLOps projects per repository.
These properties are based on the SLA from the user request. 

FLOps uses the concepts of Applications and Services to manage dependent components and concepts.
Each app can have multiple services.
Services are bound to parent apps, they cannot exist on their own.
Apps and services can and have to be created via the orchestrator as usable components.
Applications themselves are collectors of information and metadata, they do not run or contain any executable code, images, or similar.
Services are the computational components that can be deployed and undeployed.
This split is based on Oakestra's applications and services.
The two main FLOps app types are project-based apps and customer-facing ones.
The Observatory app is a customer-facing app.
There is exactly one observatory app for each user.
Users can have multiple projects.
The observatory hosts the tracking server and project observer services.
The tracking server service is used to track the projects and individual FL experiments.
It hosts the GUI. 
(It utilizes the MLFlow tracking server mentioned in \ref{subsection:mlflow}.)
When users request/start a new project the observatory is created with all its components in case it did not yet exist.
Users can request access to the GUI/tracking-server independently from a project.
A Project Observer service is used to gather and show information or updates regarding the project status to the user.
The project observer informs the user if there are any issues during the project live time, such as dependency issues during the containerizated image builds.
There is one Project Observer per project to improve readability and comprehension.

\begin{figure}[h]
    \centering
    \includegraphics[width=1.0\textwidth]{uml_analysis_object_model_repo.png}
    \caption{FLOps ML Code Repository UML Analysis Object Model}
    \label{fig:uml_repo_analysis_object_model}
\end{figure}

Figure \ref{fig:uml_repo_analysis_object_model} shows additional details of the ML Code Repositories from the core model.
Users can provide a link to ML code repositories for FLOps to augment and train.
For this to be possible and straightforward the repository needs to fulfill the following structural requirements.
The repository needs to have a dedicated file that lists all necessary dependencies to train its model.
Theoretically it should be possible to extract these requirements dynamically by inspecting the code.
However, this is a complex and error prone endeavour.
To avoid this issues users should provide the dependencies they used for training.
We recommend to run the training locally on some exemplary or mock data and record the dependencies via MLflow's auto logging functionality.
This is a possible and easy approach to get a suitable dependency file.
Note that this is not a guarantee that the dependencies will be compatible, because MLflow's dependency logging can be erroneous.
Before providing the dependency file to FLOps we recommend to make sure the dependencies are sufficient and compatible.

For FLOps to augment the ML code and utilize is properly FLOps excepts the repository to implement a model manager and data manger.
The model manager is the interface to access the model and its data and parameters.
It further allows to train and evaluate the model.
It calls its linked data manager to prepare the data and retrieve it once it is ready.
The data itself should not be part of the repository.
The prepare data method will call a FLOps method that will be added during FL augmentation.
The user has to define in prepareData how to preprocess the retrieved data for individual training.

Both managers have an abstract parent class that users can import during implementation for guidelines.
These templates can be found as part of the FLOps Utils pip package \cite{flops_utils_pip}.

\begin{figure}[h]
    \centering
    \includegraphics[width=1.0\textwidth]{uml_analysis_object_model_project.png}
    \caption{FLOps Project UML Analysis Object Model}
    \label{fig:uml_project_analysis_object_model}
\end{figure}

Figure \ref{fig:uml_project_analysis_object_model} shows further details about a FLOps project's contents.
Users can customize their projects via the SLA that is included in their API requests.
One possible customization is to specify resource constraints such as memory or storage.
Users can customize the FL training by changing the project's training configuration.
The same ML repository can be trained differently depending on these configurations.
This configuration includes a mode that tells FLOps to perform different types of FL if applicable.
Currently FLOps supports classic and (clustered) HFL.
Only training data will be used that matches the provided data tags.
The training rounds configure the number of training and evaluation rounds that each learner performs.
The training cycles are only used for HFL.
The training rounds mean the number of training rounds on performed on each learner per cycle.
A training cycle stands for the number of training rounds between the root and cluster aggregators, which can be seen as aggregator and learners, thus classic FL.
For example if the user requests 3 cycles and 5 rounds this means that the learners will train 5 rounds per cycle for 3 cycles.
In total each learner will train for 15 rounds during the entire project runtime.
The depicted attributes are only a supset of currently available and possible configurations.

\begin{figure}[h]
    \centering
    \includegraphics[width=1.0\textwidth]{uml_analysis_object_model_project_services.png}
    \caption{FLOps Project Services UML Analysis Object Model}
    \label{fig:uml_project_services_analysis_object_model}
\end{figure}

The core figure \ref{fig:uml_core_analysis_object_model} only alluded that a project consists out of several services and depicted only the project observer.
Figure \ref{fig:uml_project_services_analysis_object_model} expands upon this and shows important project services and their relationships.
There are three main project services.
The FL Image Builder is a service that builds containerized images.
It can build the FL augmented images for the learner and aggregator as well as the inference server of the trained model.
This distinction is made via the buildPlans.
The builder clones the ML repository, handles and checks the provided dependencies, builds the images and pushes them to an image registry.
During and and the end of the builder operation the service notifies other components including the project observer about its progress, current state and potential errors.

The FL Aggregator manages the FL training loop and holds the global Model and strategy for training.
It starts its internal FL server for learners to register for training.
The aggregator starts and terminates learning rounds and cycles
It logs results like metrics or the final trained model via the tracking server.
Similarly to the builder it notifies other components during runtime about its progress and errors.

The aggregator and learners utilize the code provided in the user's ML code repositories.
They have direct access to the model and data managers.
Both are injected via the image builder.

The FL learners are project services that perform the FL training on local data.
They fetch locally stored data, connect to the aggregator, and perform FL activities such as training.
The learner uses the code found in the model and data managers and wrap itself around their implemented interface methods.
As a results users do not need to implement the FL (boilerplate) code themselves.
Therefore, a getParameters methods of a learner uses the getParameters method described in the user's ML repository with additional logic around it.
Learners also notify other components about their progress or failures.

\begin{figure}[h]
    \centering
    \includegraphics[width=0.55\textwidth]{uml_analysis_object_model_aggregators.png}
    \caption{FLOps Aggregator Types UML Analysis Object Model}
    \label{fig:uml_project_aggregators_analysis_object_model}
\end{figure}

Figure \ref{fig:uml_project_aggregators_analysis_object_model} shows the simplified relation between different FLOps aggregator types.
Because FLOps support classic and hierarchical FL it needs to support different aggregator types.
For conventional FL the classic aggregators are used, whereas for HFL one root aggregator is created with one cluster aggregator per cluster available in the orchestrator.
The root orchestrator sees cluster orchestrators as plain learners.
A cluster aggregator is a hybrid between an aggregator and a learner.

\subsection{Dynamic Model}
%TODO paraphraze
"The dynamic model focuses on the behavior of the system. The dynamic model is
depicted with sequence diagrams and with state machines. Sequence diagrams represent the
interactions among a set of objects during a single use case. State machines represent the
behavior of a single object (or a group of very tightly coupled objects). The dynamic model
serves to assign responsibilities to individual classes and, in the process, to identify new classes,
associations, and attributes to be added to the analysis object model."


\begin{figure}[h]
    \begin{adjustwidth}{-0.2\paperwidth}{-0.2\paperwidth}
        \centering
        \includegraphics[width=1.0\paperwidth]{uml_sequence_init.png}
        \caption{TODO}
        \label{fig:uml_sequence_init}
    \end{adjustwidth}
\end{figure}

\begin{figure}[h]
    \begin{adjustwidth}{-0.2\paperwidth}{-0.2\paperwidth}
        \centering
        \includegraphics[width=0.95\paperwidth]{uml_sequence_diagram_project_start.png}
        \caption{TODO}
        \label{fig:uml_sequence_project_start}
    \end{adjustwidth}
\end{figure}

\begin{figure}[h]
    \begin{adjustwidth}{-0.2\paperwidth}{-0.2\paperwidth}
        \centering
        \includegraphics[width=0.95\paperwidth]{uml_sequence_actor_builder.png}
        \caption{TODO}
        \label{fig:uml_sequence_project_start}
    \end{adjustwidth}
\end{figure}


