\section{Requirements Elicitation \& Specification}

It uses the problem statement \ref{section:problem_statement}, objectives \ref{section:objectives}, and weaknesses found in the FL field \ref{subsection:fl_research} as input to elicit requirements.
These requirements are specified, analyzed, and concretized. 
This chapter starts by analyzing and deriving functional and nonfunctional requirements.
It utilizes scenarios, use cases, and (UML) models to explain the proposed system's functionalities, dynamics, and structure.
The contributions section \ref{section:contributions} provides an overview of FLOps's functionalities and features and how they realize the discussed objectives.
This chapter follows the Requirements Analysis Document Template by Brügge et al. \cite{book:bruegge}.

\input{chapters/requirements_engineering_and_system_design/subsections/functional_requirements.tex}
\input{chapters/requirements_engineering_and_system_design/subsections/nonfunctional_requirements.tex}
