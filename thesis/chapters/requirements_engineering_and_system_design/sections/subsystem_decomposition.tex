\subsection{Subsystem Decomposition}
This subsection provides a more detailed look at FLOps' overall architecture and its subsystems.
This is a concretion of the system overview seen in Figure \ref{fig:flops_structure_overview}.
Decomposing a large system into its sub-components provides new insights and improves comprehension of the system.
The general approach for this endeavor is to use a UML component diagram.
It allows to discover if the system follows teh openclosed principle of good software design.
This principle states that a system should "stay open for extension, but closed for modification" \cite{book:bruegge}.
To achieve this the system should strive for minimizing coupling and maximizing cohesion.
Cohesion expresses how tightly components of the same subsystem work together.
Coupling describes how directly dependent components from different subsystems are on each other without utilizing unifying interfaces, access points, or facades.

\begin{figure}[p]
    \begin{adjustwidth}{-0.175\paperwidth}{-0.175\paperwidth}
        \centering
        \includegraphics[width=0.975\paperwidth]{uml_component_diagram_overview.png}
        \caption{FLOps Subsystem Decomposition}
        \label{fig:component_diagram_overview}
    \end{adjustwidth}
\end{figure}

Figure \ref{fig:component_diagram_overview} shows a UML component diagram of the major FLOps components and subsystems.
The largest subsystems are the orchestrated layer and the FLOps management.
The dotted lines in the diagram depict dependency relationships between components and subsystems.
The arrow point towards a dependency.
Some of these dotted lines could be replaced with provided and required interfaces (lollipop notation).
We decided to use dotted lines instead to reduce complexity.
The little squares where interfaces come out of are so called UML component ports.
They represent concrete interaction points.
We use them to signify that a concrete network port is well known.
Interfaces without ports have unconventional, unknown, or complex access points.

The FLOps management consists out of six components.
The backend and artifact stores keep training metrics and models respectively.
They are MLflow components.
The FLOps database stores all persistent information about FLOps' projects, components, apps, and services.
The FLOps manager coordinates all FLOps processes.
These processes include, serving a RESTful API for user requests, coordinating deployments with the orchestrator, checking requirements via its image registry, and accessing the user's repository.
The manager is by far the most important and largest part of the FLOps management suite.
The MQTT broker enables lightweight communication between the manager and deployed FLOps services.
The FLOps image registry provides full control of and direct access to images built by the image builder services.

The orchestrated layer contains the control plane and the worker nodes.
The control plane is independent of FLOps.
Only the FLOps manager interact with it via its own REST API.
The control plane resolves the FLOps management requests and creates, (un)deploys, or deletes components necessary for FLOps to run.
The relevant apps and services for FLOps are deployed on a single or multiple worker nodes.
The two key FLOps apps are the project and observatory.

All project services can share their status with the manager or project observer over MQTT or socket messages.
The three services are dependent on the FLOps image registry.
The image builder needs to push his images to it, while the FL actors are pulled from it.
By default, the learners and aggregator(s) communicate via gRPC (Flower).

The observatory services are the project observer(s) and tracking server.
The tracking server hosts the HTTP based GUI and is accessible over a REST API.
The aggregator sends his logged metrics and model over the tracking server to the management stores.
The user can interact with the system via the GUI or by accessing the event (logs) of the project observer over the control plane API or FLOps CLI.

The management image registry, project's image builder and both observatory services are pulled from FLOps' public git image registry \cite{flops_code}.