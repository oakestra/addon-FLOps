
\subsection{DevOps}

Older methods like the waterfall model split up the development and operations tasks and involved individuals.
Software was first developed by one team and then operated by another.
Due to the massive increase and modern requirements for flexibility and ability for change,
developmental and operational tasks now form an interconnected infinite loop.
For example, a company develops the first version of a software product in-house.
To distribute this software among their clients and make it accessible, they
build distributable software artifacts based on their source code.
These artifacts might be container images or executable binaries.
They publish these artifacts to online registries and roll live services out in the cloud.
Users enjoy this product and request further features.
The loop starts anew.
The new features lead to unexpected bugs.
The loop starts again, and so on.
A software loop is only as fast as its slowest step.

In today's world, this loop is rarely a linear set of steps but several ones.
Such loops are running in parallel at different stages several times per day.
This concurrency is especially noticeable in projects that divide software into multiple decoupled parts.
For example, in micro-service architectures, one service might be buggy and need fixing,
while another is receiving a feature update.
These dynamic and strong dependencies require developmental and operational tasks
to work tightly together.
This coupling also applies to IT professionals that need to
cooperate and understand each other's areas well.
This combined effort has become its own broad disciple called DevOps.

Due to this synergy, new techniques, tools and professions arose 
for various tasks, like building, deploying, testing, and monitoring.
One core activity in this connected discipline is automation,
because repetitive manual labor is an inefficient and expensive bottleneck.
Prominent tools include Ansible and Gitlab-CI/CD.
DevOps is a very broad discipline without concrete borders,
so the activities of building artifacts or container images, orchestration, or knowledge sharing can be considered as part of DevOps.
This notion would make Git, Docker, and Kubernetes the primary tools in this field.

An essential concept in DevOps is CI/CD, which stands for continuous integration, continuous delivery, and deployment.
CI/CD focused on automating this software loop via custom pipelines.
A DevOps pipeline is comparable to an assembly line in a factory.
A software product needs to pass several connected stages with multiple steps.
These stages can include testing, building, releasing, and deployment.

DevOps as a term was first mentioned around 2009 \cite{paper:mlops}, yet it is still a very active and rapidly evolving field
that unfortunately many other disciplines are not taking inspiration from or taking advantage of.
