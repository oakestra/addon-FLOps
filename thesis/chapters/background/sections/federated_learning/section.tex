\section{Federated Learning}\label{section:federated_learning}

This section contains necessary background information and context regarding FL.
The first subsection covers fundamental FL building blocks and terminologies.
The next subsection explains vital supplementary FL concepts.
Oakestra orchestrates FLOps.
It uses an unconventional three-tiered structure that allows support for geographical clusters \cite{paper:oakestra_usenix}.
This structure opens up unique opportunities for FL applications.
More advanced FL architectures are necessary to benefit from these opportunities.
The following subsection discusses these architectures.
Building on top of this solid FL understanding, the subsequent subsection reviews the research landscape of FL.
That subsection showcases active and popular research directions.
It points out underexplored aspects and weak points in the field.
The penultimate subsection analyzes and compares existing FL frameworks and libraries.
The concluding subsection provides an overview of the FL framework FLOps uses.

% also talk briefly about FL history (that it is rather young)
% FL - Basics (as a solution for laws & privacy)
% from ML -> to FL
% what is FL
% basic terminologies
% "classic" FL
% classic algo
\subsection{Basics}
\subsection{Supplementary FL Concepts}

In this subsection we explore important supplementary FL concepts
to get a better understanding of the field.

\subsubsection{FL compared to Distributed Learning}

On first glance FL seems similar to Distributed Learning (DL).
Both are used for computationally expensive large ML tasks.
To avoid the need for one extraordinary powerful machine,
and increase convergence times, 
the computations are distributed among many weaker machines that
train individually. Afterward, a global model gets aggregated
at the server.

Now to the differences.
In FL, the quantity and distribution of training data
can be very diverse and might stay unknown for the entire FL process.
FL only uses the data that the learners offer. 
DL starts with full centralized access and control to the entirety of data,
before splitting it up among its fixed and predefined clients.
Thus, DL does not support the privacy concerns because it has
total oversight and control of all data and how it should be split up.
In FL the data might be IID or non-IID, different learners can have
different amounts of data.
The number of learners in FL can be very dynamic.
Some devices might only join for a few training rounds,
or crash/fail/disconnect during training.

\subsubsection{FL Variety}

The goal for this part is to showcase that FL can
be used for various different ML paradigms and use cases.

As we have already discussed in the first FL subsection,
FL can train DNNs.
FL is also applicable for classical ML models, such as
linear models (logistic regression, classification, and more) or
decision trees for explainable classifications.
Plentiful FL optimizations, such as custom algorithms and strategies,
exist for each of these ML variants.

FL can also support horizontal, vertical, and split learning.
Horizontal learning can be useful in scenarios where the available data features are the same but
originate from different sources.
For example, patient data from different hospitals
that records the same features like age, and ailment.
Vertical learning is useful when different data samples have
different feature spaces.
In the hospital example, this would mean asking different doctors/experts
about the same patients. 
The patient reports would be about the same individuals but include different features,
like cardiological metrics or neurological metrics.
We omit to discuss split learning due to its complexity that would bloat this thesis.

In case the global model is too general and does not satisfy the individual needs
of a learner, personalization can be employed.
Different personalized FL (PFL) approaches exist.
Some take the final trained global model and further train it on local data (fine tuning).
Other techniques train two local models concurrently.
One model that gets shared and updated.
The second one stays isolated and only get influenced by local data.
For inference a mixture between the global and purely local model can be used.
PFL is a deep and growing subfield of FL.


\subsubsection{FL Security \& Privacy}
Secure FL should use secure and authenticated
communication channels to avoid messages to be intercepted,
read, or impersonated via a man in the middle adversary.
To help with that one should ensure that 
learners and aggregators are the only actors that have
access to those messages and can decipher them.
There are two kinds of adversaries in FL.
Insiders that are part of the FL process, such as
malicious aggregators or learners.
Or outsiders that try to interfere from beyond the FL system.

A variety of FL threats exist.
One example is manipulation where insiders try to distort
the model to their own advantage by tinkering with FL components
that the attacker has access to.
The attack goals include polluting the global model
to misclassify (Backdoor).
If the attack is un-targeted (Byzantine), by injecting random noise,
or flip labels, the model performance can degrade.
It is difficult to detect malicious activity because
FL can support dynamic or even unknown numbers of learners
that can use vastly different non-IID data.
It can be unclear if the learner is innocent and simply has
access to unusual data, or if the learner is adversarial.
Another example is if there are no safeguards in place during aggregation,
a malicious learner can claim to have used an overwhelming amount of training samples,
thus overshadowing other participants and influencing the global model the most.
As a result, even very scarce, well timed attacks in FL can have devastating impact.

Another threat comes from inference, where 
insiders or outsiders try to extract sensitive information
about the used training data.
In classical FL leakage of privacy can only occur via inference.
Inference attacks try to deduce private information from
artifacts that the FL process produces.
A large body of ML research exists that is dedicated
to analyzing and protecting against such attacks.
There are different subtypes of inference attacks.
One example is the membership attack that tries to find if
specific samples were used for training or not.
Another attack is called extraction attack, which tries to
obtain all training samples.
The challenge here is that attackers have easy access to the final model.
Malicious insiders can even attack intermediate models,
Model inversion attacks are different attack variant,
where adversaries query the trained model in peculiar ways to
reverse engineer data samples. 
If this attack gets successfully repeated it is possible to
deduce the original dataset.
Other attacks require malicious aggregators, that can
trace back which update parameters what learner provided
before aggregating the global parameters.

Fortunately, there exists a growing array of defenses against those threats.
It is important to pick and combine these defenses wisely based on the use case and environment.
One major technique is differential privacy (DP).
DP is a complex mathematical framework which is formally proven to work.
It can be used as noise for the dataset or (inference) query.
The downside is that DP might reduce the model accuracy significantly.

Secure aggregation is a prominent protection against model inversion attacks.
It securely combines individual model parameters into global ones, before
sending it to the aggregator, which makes re-engineering and backtracking a lot harder. \cite{paper:cluster-based-secure-aggregation-FL}
\subsection{FL Architectures}

FL comes in two broad structural categories.
Cross-silo or enterprise FL gets used in large data centers or multinational companies.
Each learner represents a single institution or participating group.
There are only around ten to a few dozen learners involved.
Cross-silo FL considers the identity of the parties for training and verification.
Generally, every individual local update from every learner at every training round is significant.
Fallouts and failures of individual learners are serious.

Cross-device FL can include hundreds or millions of devices, primarily edge/IoT devices.
One can say that cross-device is the opposite of cross-silo.
Due to this great pool of learners, a subset typically gets used per training round.
The identities of the participating learners are usually unimportant and get ignored.
Due to the nature of these devices and their environments, cross-device FL
needs to manage challenges, such as non-IID data, heterogeneous device hardware,
different network conditions, learner outages, or stragglers.
Various techniques exist to navigate these challenging conditions,
including specialized algorithms for aggregation or learner selection.
These strategies can consider bias, availability, resources, and battery life.
FLOps focuses on cross-device FL.
From now on, when we mention FL, we mean cross-device FL.

As discussed, FLOps wants to benefit from the unique three-tiered Oakestra \cite{paper:oakestra_usenix} architecture.
Different FL architectures exist to support such large-scale FL environments.
The two main challenges for such scenarios are managing a massive number of connections and aggregations
and reducing the negative impact of straggling learner updates.
The problem with using a single aggregator, as seen in \ref{fig:basic_fl_intro}, is
that this single aggregator becomes a communication bottleneck.
Additionally, per-round training latency is limited by the slowest participating learner.
Thus, stragglers turn into another bottleneck.
We discuss four main architectures for large-scale FL.

\subsubsection{Clustered FL}

\begin{figure}[h]
    \centering
    \includegraphics[width=\textwidth]{clustered_fl.png}
    \caption{Clustered FL Architecture}
    \label{fig:clustered_fl}
\end{figure}
Figure \ref{fig:clustered_fl} shows the Clustered FL (CFL) architecture
that groups similar learners into clusters.
CFL can base clusters on local data distribution, training latency,
available hardware or geographical location.
The issue of the singular aggregator as a bottleneck persists.
The main challenge for CFL is choosing a suitable clustering strategy
and criteria for the concrete use case.
If the criteria are very biased, the risk arises that updates from preferred clusters
will be heavily favored, resulting in a biased global model with bad generalization.
Another task is to properly profile the nodes to match them to the correct cluster.
For example, the entire cluster suffers if a slow outlier is present in a cluster.
Node properties can vary over time, so cluster membership has to be dynamic.
One should not overdo profiling.
Otherwise, privacy might get compromised.

The benefits of CFL are its ease of implementation,
familiar architecture to classic FL,
and flexibility to tune clustering/selection dynamically.
One can combine CFL with other architectures.
A downside of CFL is that a proper clustering strategy is 
use-case-dependent and challenging to optimize.
CFL does not really solve scalability issues on its own,
especially since the clustering overhead becomes critical with larger numbers of nodes.

\subsubsection{Hierarchical FL}
\begin{figure}[h]
    \centering
    \includegraphics[width=\textwidth]{hfl_architecture.png}
    \caption{Hierarchical FL Architecture}
    \label{fig:hfl_architecture}
\end{figure}
Figure \ref{fig:hfl_architecture} depicts the hierarchical FL (HFL) architecture.
In HFL, the root aggregator delegates and distributes the aggregation task to 
intermediate aggregators.
Note that HFL can have multiple layers of intermediate aggregators.
Each intermediate aggregator and its connected learners resemble an instance of classic FL.
After aggregating an intermediate model, the intermediate aggregators send their parameters
upstream to the root aggregator.
The root combines the intermediate parameters into global ones and sends them downstream for further FL rounds.

This structure requires significant modifications to the underlying FL architecture.
The proper design and implementation, as well as the assignment of learners to aggregators,
determine the success of one's FL setup.
For example, if too many learners are attached to a given aggregator, that aggregator becomes a bottleneck.
If too few learners are assigned, the intermediate aggregated model can get
very biased, and the infrastructure resource and management costs become unjustified for the small number of learners.
A management overhead arises with more components, including handling fault tolerance,
monitoring, synchronizing, and balancing.
Bad synchronization can amplify straggler problems.
Balancing refers to combining and harmonizing intermediate parameters to
get a good global model.

The benefits of HFL are its dynamic scalability and load balancing.
One can easily add or remove intermediate aggregators and their connected learners.
Due to this distribution of load and aggregation, each aggregator, including the root,
is less likely to face bottleneck issues.
One can combine HFL with CFL, where each intermediate aggregator is responsible
for one or multiple clusters.
The downsides of HFL are communication and management overheads.
More components lead to more transmitted messages.
These messages all need to be secured and encrypted.
With more components and nodes, adversaries can take advantage of more possible backdoors.

\subsubsection{Decentralized FL}
Decentralized FL does not require a central aggregator.
Instead, it operates on a peer-to-peer basis via a blockchain.
That way, the centralized communication bottleneck gets resolved.
The blockchain represents the global model.
Learners train in parallel.
Each locally trained update gets a version.
Based on this version, random clients are chosen for aggregation.
The results get appended to the blockchain, and the model version is incremented.
FLOps does not use this kind of FL, so we keep this part short.

\subsubsection{Asynchronous FL}
This architecture allows learners to train continuously and push
their updates to the aggregator once they are finished.
This method eliminates stragglers and dropout problems because
a training round does not need to wait or handle outliers and timeouts.
A new issue of staleness arises when updates are merged into the global model
that took a very long time to complete.
Such an update used a now outdated version of the global model.
As a result, the global model is partially reverted to an older state.
Asynchronous FL can be combined with other architectures.

\subsection{FL Research}\label{subsection:fl_research}

\begin{figure}[h]
    \centering
    \includegraphics[width=0.8\textwidth]{fl_documents_research.png}
    \caption{Evolution of FL Publications}
    \label{fig:fl_documents_research}
\end{figure}

Figure \ref{fig:fl_documents_research} shows the exponential growth of FL documents since 2016.
(This data comes from searching for "federated learning" in article title, abstract, or keywords via Scopus \cite{scopus_homepage}.)
The idea for this graph is based on \cite{thesis:tum_fl_framework_comparison}.
Graph \ref{fig:fl_documents_research} uses a different query with the latest available data.

Before creating FLOps, we looked for research gaps in the fields of ML at the edge, specifically FL.
We have read and examined 47 papers in detail, with 26 papers focusing on FL. 
Additionally, we consulted several articles, joined and participated in discussion forums, and completed a couple of paid courses.
Discussing each paper in detail would heavily bloat this thesis.
This subsection presents key and meta-findings instead.
While working through the material, we created and incrementally updated a database in which we noted specific properties of each paper.
These properties include one or multiple categories in which the paper fits in.
Additional properties include the initial problems or challenges the authors tried to resolve, their contributions, results, limitations, and envisioned future work.
We also noted what ML or FL frameworks or libraries they claimed to use.

\begin{figure}[p]
    \input{tables/fl_research_table_1.tex}
\end{figure}

\begin{figure}[p]
    \input{tables/fl_research_table_2.tex}
\end{figure}

\begin{figure}[p]
    \input{tables/fl_research_table_3.tex}
\end{figure}

Tables \ref{table:fl_research_table_1}, \ref{table:fl_research_table_2}, and \ref{table:fl_research_table_3} depict our analyzed FL papers.
They present the documented contributions, limitations, and future work properties.
When there is no content (-) in the "Limitations \& Future Work" column that means that the authors did not mention any explicitly and that we did not notice anything specifically.
These tables provide a good impression of the examined FL papers.
Patterns and trends can be extracted from these papers based on the documented properties.

Patterns and trends help to better understand the research field of FL as a whole.
Figure \ref{fig:fl_research_categories} shows the different categories and their distribution.
Most examined papers focused on performance, trying new concepts, finding best practices, and exploring different FL architectures.
Only two papers focused on deployment and orchestration.

\begin{figure}[p]
    \centering
    \includegraphics[width=1.0\textwidth]{fl_research_categories.png}
    \caption{FL Paper Categories}
    \label{fig:fl_research_categories}

    \includegraphics[width=1.0\textwidth]{fl_research_problem_challenge.png}
    \caption{Targeted Problems \& Challenges of FL Papers}
    \label{fig:fl_research_problem_challenge}
\end{figure}

Figure \ref{fig:fl_research_problem_challenge} reveals a similar trend.
The primary focus is on investigating new concepts or improving existing performance, scalability, and complexity bottlenecks.
Several papers have aimed to narrow the gap between industry and research or make FL easier to use.
This ease of use seems to focus on improving already configured and working FL setups.
The main contributions seen in Figure \ref{fig:fl_research_contributions} strengthen this assumption.
Mathematical and conceptual proofs dominate this chart.
They prove that novel architectures and algorithms work as proposed.
Contributions do not seem to focus on improving the initial setup, deployment, and configuration processes.

\begin{figure}[p]
    \centering
    \includegraphics[width=1.0\textwidth]{fl_research_contributions.png}
    \caption{FL Paper Contributions}
    \label{fig:fl_research_contributions}

    \includegraphics[width=0.9\textwidth]{fl_research_achieved_results.png}
    \caption{Achieved Results of FL Papers}
    \label{fig:fl_research_achieved_results}
\end{figure}

The achieved results mirror previous findings.
Figure \ref{fig:fl_research_achieved_results} shows that these contributions lead to more efficient FL.
Improved aspects include speed, resource utilization, training results, and handling of heterogeneous data.

\begin{figure}[h]
    \centering
    \includegraphics[width=0.9\textwidth]{fl_research_limitations_future_work.png}
    \caption{Limitations \& Future Work of FL Papers}
    \label{fig:fl_research_limitations_future_work}
\end{figure}

Figure \ref{fig:fl_research_limitations_future_work} reflects this perception.
If specified, the focal point is on improving privacy and security, further performance optimizations, or adding support for more ML use cases.
Even the future focus is not on optimizing accessibility, usability, or the mentioned initial vital steps.
These documented properties might be biased, and the inspected sample size of papers is relatively small.
To improve confidence in these findings, we compare them with the total number of published works about FL.

\begin{figure}[H]
    \centering
    \includegraphics[width=1.0\textwidth]{fl_publications_compared.png}
    \caption{Evolution of FL Publications based on Keywords}
    \label{fig:fl_publications_compared}
\end{figure}

Figure \ref{fig:fl_publications_compared} depicts how many works have been published in FL with specific keywords that match our custom categories.
We applied the same method to gather the data as for \ref{fig:fl_documents_research}.
The global results paint a similar picture as our samples.
The most popular topics in FL are related to privacy/security, performance, or algorithms.
Only a tiny portion of FL papers focus on usability, automation, orchestration, or other initial steps.

It seems that researchers assume others to already have working FL environments.
Furthermore, they seem to motivate their readers to optimize these setups based on their findings instead of replicating and configuring such an FL setup initially.
These tendencies are visible when inspecting the ML and FL frameworks and libraries the authors mentioned they used.
The following figure is again based on our examined papers.
Figure \ref{fig:fl_research_ml_frameworks} shows that most authors did not explicitly state what ML framework or library they used for their work.
Many researchers used Pytorch and TensorFlow.

\begin{figure}[h]
    \centering
    \includegraphics[width=0.9\textwidth]{fl_research_ml_frameworks.png}
    \caption{Distribution of mentioned ML Frameworks in FL Papers}
    \label{fig:fl_research_ml_frameworks}
\end{figure}

\begin{figure}[h]
    \centering
    \includegraphics[width=0.9\textwidth]{fl_research_fl_frameworks.png}
    \caption{Distribution of mentioned FL Frameworks in FL Papers}
    \label{fig:fl_research_fl_frameworks}
\end{figure}

Figure \ref{fig:fl_research_fl_frameworks} shows that FL researchers rarely mention what FL frameworks they use for their work.
It is much more common for authors to mention what ML framework they used than what FL framework they used.
Possible reasons for this might be that ML as a field is a lot older, more sophisticated, widespread, and established.
The same applies to ML frameworks.
On the other hand, FL is a very young subfield of ML research.
FL frameworks are still in their early stages.
FL researchers might be using FL frameworks.
However, due to the framework's immaturity, the researchers might not deem it important to explicitly point out that they used them.
Another possible explanation is that FL researchers are experts in FL and can set up and configure FL from the ground up.
Either way, this lack of transparency makes reproducing or extending their work challenging, if not infeasible.
These gaps in FL research motivated the creation of FLOps.
\subsection{FL Frameworks \& Libraries}\label{subsection:fl_frameworks_and_libraries}

To better comprehend why so many researchers did not specify or use FL frameworks, we examine
the current landscape of available FL frameworks.
We will keep this discussion short because
Saidani already analyzed and evaluated FL frameworks in great detail
in his master's thesis \cite{thesis:tum_fl_framework_comparison} from 2023.
He examined FL libraries, frameworks, and benchmarks.
He found that many FL tools exist for specific niche use cases and architectures.
This is contrary to the opinions of his questioned FL practitioners and experts, who
expect FL libraries and frameworks to focus on basic FL features,
such as communication, aggregator-learner orchestration, security, and data aggregation.
Saidani found that many libraries and frameworks, most of which are not production-ready,
are still in an experimental research state.

To reduce complexity, he focused on the five most promising open-source frameworks.
For a framework to be allegeable, it had to fulfill 2/3 of the following criteria.
It needed more than one thousand starts and 350 forks on GitHub.
The interviewed experts had to mention it.
The framework had to support all major operation systems.
Because FL is rapidly evolving, we updated his findings and expanded upon them by including
the last version released, the last commit pushed, and the number of open issues in the repository.

\begin{changemargin}{0cm}{0cm}
    \centering
    \begin{tabular}{|m{2.4cm}||c|c|c|c|c|c|c|}
        \hline
            \textbf{Framework} & \textbf{Version} & \textbf{Release} & \textbf{Stars} & \textbf{Forks} & \textbf{Last Commit} & \textbf{Issues} \\
        \hline
            Pysyft \cite{fl_framework:pysyft} & 0.9.0 & two weeks ago & 9.4k & 2k & Same Day & 2
        \\
        \hline
            Tensorflow Federated \cite{fl_framework:tensorflow_federated} & 0.85.0 & two days ago & 2.3k & 578 & Same Day & 168
        \\
        \hline
            FedML \cite{fl_framework:fedml} & 0.8.9 & 11 months ago & 4.1k & 776 & 3 months ago & 118
        \\
        \hline
            Flower \cite{fl_framework:flower} & 1.10.0 & 3 weeks ago & 4.7k & 815 & Same Day & 284
        \\
        \hline
            OpenFL \cite{fl_framework:openfl} & 9.3.4 & last month & 1.9k & 426 & Same Day & 256
        \\
        \hline
    \end{tabular}
    \captionof{table}{Updated FL Framework Comparison} 
    \label{table:updated_fl_framework_comparison}
\end{changemargin}


Table \ref{table:updated_fl_framework_comparison} shows our updated FL Framework comparison.
Note that we took these stats on 16.08.2024.
These FL frameworks are in active development. 
Only FedML has not been updated for several months now.

Saidani's main original contribution was a novel FL benchmarking suite called FMLB (Federated Machine Learning Benchmark).
He developed it to evaluate and compare the mentioned FL frameworks efficiently.
His previous analysis and summary of existing frameworks were sound and helpful.
However, we are critical of his evaluation results, especially the poor performance of Flower surprised us.
We tried to replicate his experiments, but his provided code \cite{tum_fl_framework_thesis_github}
lacks instructions on how to set up this benchmark application.

We simulated the experiments with the latest official flower version of that time,
and made sure to stick as close as possible to the same experimental setup and configuration.
Our findings show very different results.
Flower manages to solve the experiment quickly and efficiently.
Our results match the verdicts of other works comparing FL frameworks, such as \cite{comparative_analysis_of_fl_frameworks} or \cite{comprehensive_study_fl_frameworks_degree_project}.
\cite{comparative_analysis_of_fl_frameworks} is the latest work that compares FL frameworks that we considered,
and its verdict is that Flower even outperforms all its competition.

We decided to use Flower as the FL framework for FLOps.
% ~1 page explanation what Flower is - how it works - and why we choose to use it
% important - mention cons of flower - and limitations (e.g. lack of HFL, MLOps tooling, containerization/orchestration - mention github example where they tell the user to create a docker image from scratch)
\subsection{Flower}
