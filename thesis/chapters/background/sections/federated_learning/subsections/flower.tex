\subsection{Flower}

In this subsection we want to provide an overview about our FL framework of choice, Flower
and highlight important aspects that FLOps relies upon.
This open source framework has a corresponding 2022 research paper \cite{paper:flower}.
Flower's first release (0.10.0) was published in November 2020
and its first major release (1.0.0) was published in 2022 \cite{fl_framework:flower}.
For our goals we need an FL framework that is very flexible.
One major target in Flower's paper was to narrow the gap between research and production,
by allowing researchers to run high performance FL simulations and rapidly transition
to tangible production environments all via the same tool.
Another focal point of the paper was scale and parallelization.
Thus, flower offers a mature set of FL simulation techniques.
Flower supports all major operating systems, containerization, and ML libraries.
It promises to be easily customizable and extendable via a programming language and ML framework agnostic source design.
Flower strives to offer all major FL features, such as support for different data types and distributions,
pre-implemented popular FL algorithms, support for vertical and horizontal data splitting,
traditional ML tasks, like regression or clustering, DNNs, LLMs, and security mechanisms, like secure aggregation.
It enables to do FL via CPUs or GPUs.
Flower has support for various FL variants, including
PFL, edge computing, cross-silo, and cross-device.
Flower handles and implements core FL components,
but it does not handle many other aspects, like deployment, orchestration,
dependency management, or containerization.
It provides access to interfaces to ensure customizability and extendability.
Flower users can easily change and add functionality to the framework.
The heart of flower is its strategy concept.
Users can pick from more than 20 existing strategies or they can extend from basic strategies
and develop their own behavior.
The default communication protocol is gRPC, which can be exchanged.

Flower has a lot to offer, but it still has its limits.
It has no native out-of-the-box support for model pruning,
advanced security/privacy techniques, CFL, HFL, MLOps, or orchestration.

Due to Flowers flexible design users can implement their own additions and strategies
based on the available basic Flower components and realize many of these features.

On top of that flower has a modern, user-friendly, growing eco system.
A dedicated sub-project called Flower Datasets is part of this eco system.
This project is still in its infancy (v0.3.0).
It allows to easily pull HuggingFace datasets and split them up into FL optimized data fragments.
Users can configure how to split this data up.
In that way Flower Datasets allows to use common non-federated homogeneous/IID datasets
to be turned into challenging, federated, non-IID data, ideal for FL research and development.

This eco system includes,
a well structured and rich homepage,
an extensive set of tutorials, guides, example projects, and documentation that range from beginner-friendly to advanced,
The flower team has a solid and growing connection to the public and its user base.
They have open monthly community events, yearly summits, a blog, a dedicated questions and answers forum,
a slack space, and youtube channel.

It is very easy to setup flower and start working with it.
Flower can be installed via python's default package manager pip.
One needs to define the server/aggregator, strategy, and clients/learners.
The simplest cases can be implemented with a couple of dozen lines of python code.

The crucial part is to properly configure the client.
One needs to create a client class that extends from a Flower client class
and implement four essential methods.
These methods need to get and set the model parameters,
train/fit and evaluate the model.