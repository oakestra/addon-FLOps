\chapter{Background}\label{chapter:background}

As mentioned in the contributions section, FLOps combines and uses a large set of technologies from
different disciplines. To properly understand FLOps as a whole and why it combines these techniques,
it is necessary to analyze them individually.
This enables us to form a common understanding, including critical background knowledge of their benefits and downsides.
Only afterward does it make sense to discuss how FLOps merges them to create something new.

This background chapter provides a general overview for each sector and discusses aspects that
are necessary for FLOps in greater detail.
We start with exploring the field of federated learning.
FL is the core task at hand that FLOps aims to optimize.
A thorough understanding of this discipline is required to figure out where it has shortcomings.
To improve upon these weaknesses, we study the established set of 
best practices from DevOps and MLOps.
Techniques like automation and CI/CD require infrastructure and resources.
Orchestration allows us to provision, manage, and deploy such infrastructure and resources.
We review the field of orchestration technologies and provide a short
overview of Oakestra \cite{paper:oakestra_usenix} as the chosen platform.
In the final background section, we take a look at and compare a couple of existing pieces of work
that resemble FLOps.


% heavy use of https://www.notion.so/oakestra-team/Federated-Learning-A-Comprehensive-Overview-of-Methods-and-Applications-0092d624e64441ef9937c45fa39ff8f5?pvs=4
\section{Federated Learning}

% also talk briefly about FL history (that it is rather young)
% FL - Basics (as a solution for laws & privacy)
% from ML -> to FL
% what is FL
% basic terminologies
% "classic" FL
% classic algo
\subsection{Basics}
\subsection{FL Architectures}

FL comes in two broad structural categories.
Cross-silo or enterprise FL gets used in large data centers or multinational companies.
Each learner represents a single institution or participating group.
There are only around ten to a few dozen learners involved.
Cross-silo FL considers the identity of the parties for training and verification.
Generally, every individual local update from every learner at every training round is significant.
Fallouts and failures of individual learners are serious.

Cross-device FL can include hundreds or millions of devices, primarily edge/IoT devices.
One can say that cross-device is the opposite of cross-silo.
Due to this great pool of learners, a subset typically gets used per training round.
The identities of the participating learners are usually unimportant and get ignored.
Due to the nature of these devices and their environments, cross-device FL
needs to manage challenges, such as non-IID data, heterogeneous device hardware,
different network conditions, learner outages, or stragglers.
Various techniques exist to navigate these challenging conditions,
including specialized algorithms for aggregation or learner selection.
These strategies can consider bias, availability, resources, and battery life.
FLOps focuses on cross-device FL.
From now on, when we mention FL, we mean cross-device FL.

As discussed, FLOps wants to benefit from the unique three-tiered Oakestra \cite{paper:oakestra_usenix} architecture.
Different FL architectures exist to support such large-scale FL environments.
The two main challenges for such scenarios are managing a massive number of connections and aggregations
and reducing the negative impact of straggling learner updates.
The problem with using a single aggregator, as seen in \ref{fig:basic_fl_intro}, is
that this single aggregator becomes a communication bottleneck.
Additionally, per-round training latency is limited by the slowest participating learner.
Thus, stragglers turn into another bottleneck.
We discuss four main architectures for large-scale FL.

\subsubsection{Clustered FL}

\begin{figure}[h]
    \centering
    \includegraphics[width=\textwidth]{clustered_fl.png}
    \caption{Clustered FL Architecture}
    \label{fig:clustered_fl}
\end{figure}
Figure \ref{fig:clustered_fl} shows the Clustered FL (CFL) architecture
that groups similar learners into clusters.
CFL can base clusters on local data distribution, training latency,
available hardware or geographical location.
The issue of the singular aggregator as a bottleneck persists.
The main challenge for CFL is choosing a suitable clustering strategy
and criteria for the concrete use case.
If the criteria are very biased, the risk arises that updates from preferred clusters
will be heavily favored, resulting in a biased global model with bad generalization.
Another task is to properly profile the nodes to match them to the correct cluster.
For example, the entire cluster suffers if a slow outlier is present in a cluster.
Node properties can vary over time, so cluster membership has to be dynamic.
One should not overdo profiling.
Otherwise, privacy might get compromised.

The benefits of CFL are its ease of implementation,
familiar architecture to classic FL,
and flexibility to tune clustering/selection dynamically.
One can combine CFL with other architectures.
A downside of CFL is that a proper clustering strategy is 
use-case-dependent and challenging to optimize.
CFL does not really solve scalability issues on its own,
especially since the clustering overhead becomes critical with larger numbers of nodes.

\subsubsection{Hierarchical FL}
\begin{figure}[h]
    \centering
    \includegraphics[width=\textwidth]{hfl_architecture.png}
    \caption{Hierarchical FL Architecture}
    \label{fig:hfl_architecture}
\end{figure}
Figure \ref{fig:hfl_architecture} depicts the hierarchical FL (HFL) architecture.
In HFL, the root aggregator delegates and distributes the aggregation task to 
intermediate aggregators.
Note that HFL can have multiple layers of intermediate aggregators.
Each intermediate aggregator and its connected learners resemble an instance of classic FL.
After aggregating an intermediate model, the intermediate aggregators send their parameters
upstream to the root aggregator.
The root combines the intermediate parameters into global ones and sends them downstream for further FL rounds.

This structure requires significant modifications to the underlying FL architecture.
The proper design and implementation, as well as the assignment of learners to aggregators,
determine the success of one's FL setup.
For example, if too many learners are attached to a given aggregator, that aggregator becomes a bottleneck.
If too few learners are assigned, the intermediate aggregated model can get
very biased, and the infrastructure resource and management costs become unjustified for the small number of learners.
A management overhead arises with more components, including handling fault tolerance,
monitoring, synchronizing, and balancing.
Bad synchronization can amplify straggler problems.
Balancing refers to combining and harmonizing intermediate parameters to
get a good global model.

The benefits of HFL are its dynamic scalability and load balancing.
One can easily add or remove intermediate aggregators and their connected learners.
Due to this distribution of load and aggregation, each aggregator, including the root,
is less likely to face bottleneck issues.
One can combine HFL with CFL, where each intermediate aggregator is responsible
for one or multiple clusters.
The downsides of HFL are communication and management overheads.
More components lead to more transmitted messages.
These messages all need to be secured and encrypted.
With more components and nodes, adversaries can take advantage of more possible backdoors.

\subsubsection{Decentralized FL}
Decentralized FL does not require a central aggregator.
Instead, it operates on a peer-to-peer basis via a blockchain.
That way, the centralized communication bottleneck gets resolved.
The blockchain represents the global model.
Learners train in parallel.
Each locally trained update gets a version.
Based on this version, random clients are chosen for aggregation.
The results get appended to the blockchain, and the model version is incremented.
FLOps does not use this kind of FL, so we keep this part short.

\subsubsection{Asynchronous FL}
This architecture allows learners to train continuously and push
their updates to the aggregator once they are finished.
This method eliminates stragglers and dropout problems because
a training round does not need to wait or handle outliers and timeouts.
A new issue of staleness arises when updates are merged into the global model
that took a very long time to complete.
Such an update used a now outdated version of the global model.
As a result, the global model is partially reverted to an older state.
Asynchronous FL can be combined with other architectures.

\subsection{FL Research}\label{subsection:fl_research}

\begin{figure}[h]
    \centering
    \includegraphics[width=0.8\textwidth]{fl_documents_research.png}
    \caption{Evolution of FL Publications}
    \label{fig:fl_documents_research}
\end{figure}

Figure \ref{fig:fl_documents_research} shows the exponential growth of FL documents since 2016.
(This data comes from searching for "federated learning" in article title, abstract, or keywords via Scopus \cite{scopus_homepage}.)
The idea for this graph is based on \cite{thesis:tum_fl_framework_comparison}.
Graph \ref{fig:fl_documents_research} uses a different query with the latest available data.

Before creating FLOps, we looked for research gaps in the fields of ML at the edge, specifically FL.
We have read and examined 47 papers in detail, with 26 papers focusing on FL. 
Additionally, we consulted several articles, joined and participated in discussion forums, and completed a couple of paid courses.
Discussing each paper in detail would heavily bloat this thesis.
This subsection presents key and meta-findings instead.
While working through the material, we created and incrementally updated a database in which we noted specific properties of each paper.
These properties include one or multiple categories in which the paper fits in.
Additional properties include the initial problems or challenges the authors tried to resolve, their contributions, results, limitations, and envisioned future work.
We also noted what ML or FL frameworks or libraries they claimed to use.

\begin{figure}[p]
    \input{tables/fl_research_table_1.tex}
\end{figure}

\begin{figure}[p]
    \input{tables/fl_research_table_2.tex}
\end{figure}

\begin{figure}[p]
    \input{tables/fl_research_table_3.tex}
\end{figure}

Tables \ref{table:fl_research_table_1}, \ref{table:fl_research_table_2}, and \ref{table:fl_research_table_3} depict our analyzed FL papers.
They present the documented contributions, limitations, and future work properties.
When there is no content (-) in the "Limitations \& Future Work" column that means that the authors did not mention any explicitly and that we did not notice anything specifically.
These tables provide a good impression of the examined FL papers.
Patterns and trends can be extracted from these papers based on the documented properties.

Patterns and trends help to better understand the research field of FL as a whole.
Figure \ref{fig:fl_research_categories} shows the different categories and their distribution.
Most examined papers focused on performance, trying new concepts, finding best practices, and exploring different FL architectures.
Only two papers focused on deployment and orchestration.

\begin{figure}[p]
    \centering
    \includegraphics[width=1.0\textwidth]{fl_research_categories.png}
    \caption{FL Paper Categories}
    \label{fig:fl_research_categories}

    \includegraphics[width=1.0\textwidth]{fl_research_problem_challenge.png}
    \caption{Targeted Problems \& Challenges of FL Papers}
    \label{fig:fl_research_problem_challenge}
\end{figure}

Figure \ref{fig:fl_research_problem_challenge} reveals a similar trend.
The primary focus is on investigating new concepts or improving existing performance, scalability, and complexity bottlenecks.
Several papers have aimed to narrow the gap between industry and research or make FL easier to use.
This ease of use seems to focus on improving already configured and working FL setups.
The main contributions seen in Figure \ref{fig:fl_research_contributions} strengthen this assumption.
Mathematical and conceptual proofs dominate this chart.
They prove that novel architectures and algorithms work as proposed.
Contributions do not seem to focus on improving the initial setup, deployment, and configuration processes.

\begin{figure}[p]
    \centering
    \includegraphics[width=1.0\textwidth]{fl_research_contributions.png}
    \caption{FL Paper Contributions}
    \label{fig:fl_research_contributions}

    \includegraphics[width=0.9\textwidth]{fl_research_achieved_results.png}
    \caption{Achieved Results of FL Papers}
    \label{fig:fl_research_achieved_results}
\end{figure}

The achieved results mirror previous findings.
Figure \ref{fig:fl_research_achieved_results} shows that these contributions lead to more efficient FL.
Improved aspects include speed, resource utilization, training results, and handling of heterogeneous data.

\begin{figure}[h]
    \centering
    \includegraphics[width=0.9\textwidth]{fl_research_limitations_future_work.png}
    \caption{Limitations \& Future Work of FL Papers}
    \label{fig:fl_research_limitations_future_work}
\end{figure}

Figure \ref{fig:fl_research_limitations_future_work} reflects this perception.
If specified, the focal point is on improving privacy and security, further performance optimizations, or adding support for more ML use cases.
Even the future focus is not on optimizing accessibility, usability, or the mentioned initial vital steps.
These documented properties might be biased, and the inspected sample size of papers is relatively small.
To improve confidence in these findings, we compare them with the total number of published works about FL.

\begin{figure}[H]
    \centering
    \includegraphics[width=1.0\textwidth]{fl_publications_compared.png}
    \caption{Evolution of FL Publications based on Keywords}
    \label{fig:fl_publications_compared}
\end{figure}

Figure \ref{fig:fl_publications_compared} depicts how many works have been published in FL with specific keywords that match our custom categories.
We applied the same method to gather the data as for \ref{fig:fl_documents_research}.
The global results paint a similar picture as our samples.
The most popular topics in FL are related to privacy/security, performance, or algorithms.
Only a tiny portion of FL papers focus on usability, automation, orchestration, or other initial steps.

It seems that researchers assume others to already have working FL environments.
Furthermore, they seem to motivate their readers to optimize these setups based on their findings instead of replicating and configuring such an FL setup initially.
These tendencies are visible when inspecting the ML and FL frameworks and libraries the authors mentioned they used.
The following figure is again based on our examined papers.
Figure \ref{fig:fl_research_ml_frameworks} shows that most authors did not explicitly state what ML framework or library they used for their work.
Many researchers used Pytorch and TensorFlow.

\begin{figure}[h]
    \centering
    \includegraphics[width=0.9\textwidth]{fl_research_ml_frameworks.png}
    \caption{Distribution of mentioned ML Frameworks in FL Papers}
    \label{fig:fl_research_ml_frameworks}
\end{figure}

\begin{figure}[h]
    \centering
    \includegraphics[width=0.9\textwidth]{fl_research_fl_frameworks.png}
    \caption{Distribution of mentioned FL Frameworks in FL Papers}
    \label{fig:fl_research_fl_frameworks}
\end{figure}

Figure \ref{fig:fl_research_fl_frameworks} shows that FL researchers rarely mention what FL frameworks they use for their work.
It is much more common for authors to mention what ML framework they used than what FL framework they used.
Possible reasons for this might be that ML as a field is a lot older, more sophisticated, widespread, and established.
The same applies to ML frameworks.
On the other hand, FL is a very young subfield of ML research.
FL frameworks are still in their early stages.
FL researchers might be using FL frameworks.
However, due to the framework's immaturity, the researchers might not deem it important to explicitly point out that they used them.
Another possible explanation is that FL researchers are experts in FL and can set up and configure FL from the ground up.
Either way, this lack of transparency makes reproducing or extending their work challenging, if not infeasible.
These gaps in FL research motivated the creation of FLOps.
% briefly allude to FL in industry (1-2 paragraphs max)
\subsection{FL in Industry}
\subsection{FL Frameworks \& Libraries}\label{subsection:fl_frameworks_and_libraries}

To better comprehend why so many researchers did not specify or use FL frameworks, we examine
the current landscape of available FL frameworks.
We will keep this discussion short because
Saidani already analyzed and evaluated FL frameworks in great detail
in his master's thesis \cite{thesis:tum_fl_framework_comparison} from 2023.
He examined FL libraries, frameworks, and benchmarks.
He found that many FL tools exist for specific niche use cases and architectures.
This is contrary to the opinions of his questioned FL practitioners and experts, who
expect FL libraries and frameworks to focus on basic FL features,
such as communication, aggregator-learner orchestration, security, and data aggregation.
Saidani found that many libraries and frameworks, most of which are not production-ready,
are still in an experimental research state.

To reduce complexity, he focused on the five most promising open-source frameworks.
For a framework to be allegeable, it had to fulfill 2/3 of the following criteria.
It needed more than one thousand starts and 350 forks on GitHub.
The interviewed experts had to mention it.
The framework had to support all major operation systems.
Because FL is rapidly evolving, we updated his findings and expanded upon them by including
the last version released, the last commit pushed, and the number of open issues in the repository.

\begin{changemargin}{0cm}{0cm}
    \centering
    \begin{tabular}{|m{2.4cm}||c|c|c|c|c|c|c|}
        \hline
            \textbf{Framework} & \textbf{Version} & \textbf{Release} & \textbf{Stars} & \textbf{Forks} & \textbf{Last Commit} & \textbf{Issues} \\
        \hline
            Pysyft \cite{fl_framework:pysyft} & 0.9.0 & two weeks ago & 9.4k & 2k & Same Day & 2
        \\
        \hline
            Tensorflow Federated \cite{fl_framework:tensorflow_federated} & 0.85.0 & two days ago & 2.3k & 578 & Same Day & 168
        \\
        \hline
            FedML \cite{fl_framework:fedml} & 0.8.9 & 11 months ago & 4.1k & 776 & 3 months ago & 118
        \\
        \hline
            Flower \cite{fl_framework:flower} & 1.10.0 & 3 weeks ago & 4.7k & 815 & Same Day & 284
        \\
        \hline
            OpenFL \cite{fl_framework:openfl} & 9.3.4 & last month & 1.9k & 426 & Same Day & 256
        \\
        \hline
    \end{tabular}
    \captionof{table}{Updated FL Framework Comparison} 
    \label{table:updated_fl_framework_comparison}
\end{changemargin}


Table \ref{table:updated_fl_framework_comparison} shows our updated FL Framework comparison.
Note that we took these stats on 16.08.2024.
These FL frameworks are in active development. 
Only FedML has not been updated for several months now.

Saidani's main original contribution was a novel FL benchmarking suite called FMLB (Federated Machine Learning Benchmark).
He developed it to evaluate and compare the mentioned FL frameworks efficiently.
His previous analysis and summary of existing frameworks were sound and helpful.
However, we are critical of his evaluation results, especially the poor performance of Flower surprised us.
We tried to replicate his experiments, but his provided code \cite{tum_fl_framework_thesis_github}
lacks instructions on how to set up this benchmark application.

We simulated the experiments with the latest official flower version of that time,
and made sure to stick as close as possible to the same experimental setup and configuration.
Our findings show very different results.
Flower manages to solve the experiment quickly and efficiently.
Our results match the verdicts of other works comparing FL frameworks, such as \cite{comparative_analysis_of_fl_frameworks} or \cite{comprehensive_study_fl_frameworks_degree_project}.
\cite{comparative_analysis_of_fl_frameworks} is the latest work that compares FL frameworks that we considered,
and its verdict is that Flower even outperforms all its competition.

We decided to use Flower as the FL framework for FLOps.
% ~1 page explanation what Flower is - how it works - and why we choose to use it
% important - mention cons of flower - and limitations (e.g. lack of HFL, MLOps tooling, containerization/orchestration - mention github example where they tell the user to create a docker image from scratch)
\subsection{Flower}


% heavy use of https://www.notion.so/oakestra-team/Federated-Learning-A-Comprehensive-Overview-of-Methods-and-Applications-0092d624e64441ef9937c45fa39ff8f5?pvs=4
\section{Federated Learning}

% also talk briefly about FL history (that it is rather young)
% FL - Basics (as a solution for laws & privacy)
% from ML -> to FL
% what is FL
% basic terminologies
% "classic" FL
% classic algo
\subsection{Basics}
\subsection{FL Architectures}

FL comes in two broad structural categories.
Cross-silo or enterprise FL gets used in large data centers or multinational companies.
Each learner represents a single institution or participating group.
There are only around ten to a few dozen learners involved.
Cross-silo FL considers the identity of the parties for training and verification.
Generally, every individual local update from every learner at every training round is significant.
Fallouts and failures of individual learners are serious.

Cross-device FL can include hundreds or millions of devices, primarily edge/IoT devices.
One can say that cross-device is the opposite of cross-silo.
Due to this great pool of learners, a subset typically gets used per training round.
The identities of the participating learners are usually unimportant and get ignored.
Due to the nature of these devices and their environments, cross-device FL
needs to manage challenges, such as non-IID data, heterogeneous device hardware,
different network conditions, learner outages, or stragglers.
Various techniques exist to navigate these challenging conditions,
including specialized algorithms for aggregation or learner selection.
These strategies can consider bias, availability, resources, and battery life.
FLOps focuses on cross-device FL.
From now on, when we mention FL, we mean cross-device FL.

As discussed, FLOps wants to benefit from the unique three-tiered Oakestra \cite{paper:oakestra_usenix} architecture.
Different FL architectures exist to support such large-scale FL environments.
The two main challenges for such scenarios are managing a massive number of connections and aggregations
and reducing the negative impact of straggling learner updates.
The problem with using a single aggregator, as seen in \ref{fig:basic_fl_intro}, is
that this single aggregator becomes a communication bottleneck.
Additionally, per-round training latency is limited by the slowest participating learner.
Thus, stragglers turn into another bottleneck.
We discuss four main architectures for large-scale FL.

\subsubsection{Clustered FL}

\begin{figure}[h]
    \centering
    \includegraphics[width=\textwidth]{clustered_fl.png}
    \caption{Clustered FL Architecture}
    \label{fig:clustered_fl}
\end{figure}
Figure \ref{fig:clustered_fl} shows the Clustered FL (CFL) architecture
that groups similar learners into clusters.
CFL can base clusters on local data distribution, training latency,
available hardware or geographical location.
The issue of the singular aggregator as a bottleneck persists.
The main challenge for CFL is choosing a suitable clustering strategy
and criteria for the concrete use case.
If the criteria are very biased, the risk arises that updates from preferred clusters
will be heavily favored, resulting in a biased global model with bad generalization.
Another task is to properly profile the nodes to match them to the correct cluster.
For example, the entire cluster suffers if a slow outlier is present in a cluster.
Node properties can vary over time, so cluster membership has to be dynamic.
One should not overdo profiling.
Otherwise, privacy might get compromised.

The benefits of CFL are its ease of implementation,
familiar architecture to classic FL,
and flexibility to tune clustering/selection dynamically.
One can combine CFL with other architectures.
A downside of CFL is that a proper clustering strategy is 
use-case-dependent and challenging to optimize.
CFL does not really solve scalability issues on its own,
especially since the clustering overhead becomes critical with larger numbers of nodes.

\subsubsection{Hierarchical FL}
\begin{figure}[h]
    \centering
    \includegraphics[width=\textwidth]{hfl_architecture.png}
    \caption{Hierarchical FL Architecture}
    \label{fig:hfl_architecture}
\end{figure}
Figure \ref{fig:hfl_architecture} depicts the hierarchical FL (HFL) architecture.
In HFL, the root aggregator delegates and distributes the aggregation task to 
intermediate aggregators.
Note that HFL can have multiple layers of intermediate aggregators.
Each intermediate aggregator and its connected learners resemble an instance of classic FL.
After aggregating an intermediate model, the intermediate aggregators send their parameters
upstream to the root aggregator.
The root combines the intermediate parameters into global ones and sends them downstream for further FL rounds.

This structure requires significant modifications to the underlying FL architecture.
The proper design and implementation, as well as the assignment of learners to aggregators,
determine the success of one's FL setup.
For example, if too many learners are attached to a given aggregator, that aggregator becomes a bottleneck.
If too few learners are assigned, the intermediate aggregated model can get
very biased, and the infrastructure resource and management costs become unjustified for the small number of learners.
A management overhead arises with more components, including handling fault tolerance,
monitoring, synchronizing, and balancing.
Bad synchronization can amplify straggler problems.
Balancing refers to combining and harmonizing intermediate parameters to
get a good global model.

The benefits of HFL are its dynamic scalability and load balancing.
One can easily add or remove intermediate aggregators and their connected learners.
Due to this distribution of load and aggregation, each aggregator, including the root,
is less likely to face bottleneck issues.
One can combine HFL with CFL, where each intermediate aggregator is responsible
for one or multiple clusters.
The downsides of HFL are communication and management overheads.
More components lead to more transmitted messages.
These messages all need to be secured and encrypted.
With more components and nodes, adversaries can take advantage of more possible backdoors.

\subsubsection{Decentralized FL}
Decentralized FL does not require a central aggregator.
Instead, it operates on a peer-to-peer basis via a blockchain.
That way, the centralized communication bottleneck gets resolved.
The blockchain represents the global model.
Learners train in parallel.
Each locally trained update gets a version.
Based on this version, random clients are chosen for aggregation.
The results get appended to the blockchain, and the model version is incremented.
FLOps does not use this kind of FL, so we keep this part short.

\subsubsection{Asynchronous FL}
This architecture allows learners to train continuously and push
their updates to the aggregator once they are finished.
This method eliminates stragglers and dropout problems because
a training round does not need to wait or handle outliers and timeouts.
A new issue of staleness arises when updates are merged into the global model
that took a very long time to complete.
Such an update used a now outdated version of the global model.
As a result, the global model is partially reverted to an older state.
Asynchronous FL can be combined with other architectures.

\subsection{FL Research}\label{subsection:fl_research}

\begin{figure}[h]
    \centering
    \includegraphics[width=0.8\textwidth]{fl_documents_research.png}
    \caption{Evolution of FL Publications}
    \label{fig:fl_documents_research}
\end{figure}

Figure \ref{fig:fl_documents_research} shows the exponential growth of FL documents since 2016.
(This data comes from searching for "federated learning" in article title, abstract, or keywords via Scopus \cite{scopus_homepage}.)
The idea for this graph is based on \cite{thesis:tum_fl_framework_comparison}.
Graph \ref{fig:fl_documents_research} uses a different query with the latest available data.

Before creating FLOps, we looked for research gaps in the fields of ML at the edge, specifically FL.
We have read and examined 47 papers in detail, with 26 papers focusing on FL. 
Additionally, we consulted several articles, joined and participated in discussion forums, and completed a couple of paid courses.
Discussing each paper in detail would heavily bloat this thesis.
This subsection presents key and meta-findings instead.
While working through the material, we created and incrementally updated a database in which we noted specific properties of each paper.
These properties include one or multiple categories in which the paper fits in.
Additional properties include the initial problems or challenges the authors tried to resolve, their contributions, results, limitations, and envisioned future work.
We also noted what ML or FL frameworks or libraries they claimed to use.

\begin{figure}[p]
    \input{tables/fl_research_table_1.tex}
\end{figure}

\begin{figure}[p]
    \input{tables/fl_research_table_2.tex}
\end{figure}

\begin{figure}[p]
    \input{tables/fl_research_table_3.tex}
\end{figure}

Tables \ref{table:fl_research_table_1}, \ref{table:fl_research_table_2}, and \ref{table:fl_research_table_3} depict our analyzed FL papers.
They present the documented contributions, limitations, and future work properties.
When there is no content (-) in the "Limitations \& Future Work" column that means that the authors did not mention any explicitly and that we did not notice anything specifically.
These tables provide a good impression of the examined FL papers.
Patterns and trends can be extracted from these papers based on the documented properties.

Patterns and trends help to better understand the research field of FL as a whole.
Figure \ref{fig:fl_research_categories} shows the different categories and their distribution.
Most examined papers focused on performance, trying new concepts, finding best practices, and exploring different FL architectures.
Only two papers focused on deployment and orchestration.

\begin{figure}[p]
    \centering
    \includegraphics[width=1.0\textwidth]{fl_research_categories.png}
    \caption{FL Paper Categories}
    \label{fig:fl_research_categories}

    \includegraphics[width=1.0\textwidth]{fl_research_problem_challenge.png}
    \caption{Targeted Problems \& Challenges of FL Papers}
    \label{fig:fl_research_problem_challenge}
\end{figure}

Figure \ref{fig:fl_research_problem_challenge} reveals a similar trend.
The primary focus is on investigating new concepts or improving existing performance, scalability, and complexity bottlenecks.
Several papers have aimed to narrow the gap between industry and research or make FL easier to use.
This ease of use seems to focus on improving already configured and working FL setups.
The main contributions seen in Figure \ref{fig:fl_research_contributions} strengthen this assumption.
Mathematical and conceptual proofs dominate this chart.
They prove that novel architectures and algorithms work as proposed.
Contributions do not seem to focus on improving the initial setup, deployment, and configuration processes.

\begin{figure}[p]
    \centering
    \includegraphics[width=1.0\textwidth]{fl_research_contributions.png}
    \caption{FL Paper Contributions}
    \label{fig:fl_research_contributions}

    \includegraphics[width=0.9\textwidth]{fl_research_achieved_results.png}
    \caption{Achieved Results of FL Papers}
    \label{fig:fl_research_achieved_results}
\end{figure}

The achieved results mirror previous findings.
Figure \ref{fig:fl_research_achieved_results} shows that these contributions lead to more efficient FL.
Improved aspects include speed, resource utilization, training results, and handling of heterogeneous data.

\begin{figure}[h]
    \centering
    \includegraphics[width=0.9\textwidth]{fl_research_limitations_future_work.png}
    \caption{Limitations \& Future Work of FL Papers}
    \label{fig:fl_research_limitations_future_work}
\end{figure}

Figure \ref{fig:fl_research_limitations_future_work} reflects this perception.
If specified, the focal point is on improving privacy and security, further performance optimizations, or adding support for more ML use cases.
Even the future focus is not on optimizing accessibility, usability, or the mentioned initial vital steps.
These documented properties might be biased, and the inspected sample size of papers is relatively small.
To improve confidence in these findings, we compare them with the total number of published works about FL.

\begin{figure}[H]
    \centering
    \includegraphics[width=1.0\textwidth]{fl_publications_compared.png}
    \caption{Evolution of FL Publications based on Keywords}
    \label{fig:fl_publications_compared}
\end{figure}

Figure \ref{fig:fl_publications_compared} depicts how many works have been published in FL with specific keywords that match our custom categories.
We applied the same method to gather the data as for \ref{fig:fl_documents_research}.
The global results paint a similar picture as our samples.
The most popular topics in FL are related to privacy/security, performance, or algorithms.
Only a tiny portion of FL papers focus on usability, automation, orchestration, or other initial steps.

It seems that researchers assume others to already have working FL environments.
Furthermore, they seem to motivate their readers to optimize these setups based on their findings instead of replicating and configuring such an FL setup initially.
These tendencies are visible when inspecting the ML and FL frameworks and libraries the authors mentioned they used.
The following figure is again based on our examined papers.
Figure \ref{fig:fl_research_ml_frameworks} shows that most authors did not explicitly state what ML framework or library they used for their work.
Many researchers used Pytorch and TensorFlow.

\begin{figure}[h]
    \centering
    \includegraphics[width=0.9\textwidth]{fl_research_ml_frameworks.png}
    \caption{Distribution of mentioned ML Frameworks in FL Papers}
    \label{fig:fl_research_ml_frameworks}
\end{figure}

\begin{figure}[h]
    \centering
    \includegraphics[width=0.9\textwidth]{fl_research_fl_frameworks.png}
    \caption{Distribution of mentioned FL Frameworks in FL Papers}
    \label{fig:fl_research_fl_frameworks}
\end{figure}

Figure \ref{fig:fl_research_fl_frameworks} shows that FL researchers rarely mention what FL frameworks they use for their work.
It is much more common for authors to mention what ML framework they used than what FL framework they used.
Possible reasons for this might be that ML as a field is a lot older, more sophisticated, widespread, and established.
The same applies to ML frameworks.
On the other hand, FL is a very young subfield of ML research.
FL frameworks are still in their early stages.
FL researchers might be using FL frameworks.
However, due to the framework's immaturity, the researchers might not deem it important to explicitly point out that they used them.
Another possible explanation is that FL researchers are experts in FL and can set up and configure FL from the ground up.
Either way, this lack of transparency makes reproducing or extending their work challenging, if not infeasible.
These gaps in FL research motivated the creation of FLOps.
% briefly allude to FL in industry (1-2 paragraphs max)
\subsection{FL in Industry}
\subsection{FL Frameworks \& Libraries}\label{subsection:fl_frameworks_and_libraries}

To better comprehend why so many researchers did not specify or use FL frameworks, we examine
the current landscape of available FL frameworks.
We will keep this discussion short because
Saidani already analyzed and evaluated FL frameworks in great detail
in his master's thesis \cite{thesis:tum_fl_framework_comparison} from 2023.
He examined FL libraries, frameworks, and benchmarks.
He found that many FL tools exist for specific niche use cases and architectures.
This is contrary to the opinions of his questioned FL practitioners and experts, who
expect FL libraries and frameworks to focus on basic FL features,
such as communication, aggregator-learner orchestration, security, and data aggregation.
Saidani found that many libraries and frameworks, most of which are not production-ready,
are still in an experimental research state.

To reduce complexity, he focused on the five most promising open-source frameworks.
For a framework to be allegeable, it had to fulfill 2/3 of the following criteria.
It needed more than one thousand starts and 350 forks on GitHub.
The interviewed experts had to mention it.
The framework had to support all major operation systems.
Because FL is rapidly evolving, we updated his findings and expanded upon them by including
the last version released, the last commit pushed, and the number of open issues in the repository.

\begin{changemargin}{0cm}{0cm}
    \centering
    \begin{tabular}{|m{2.4cm}||c|c|c|c|c|c|c|}
        \hline
            \textbf{Framework} & \textbf{Version} & \textbf{Release} & \textbf{Stars} & \textbf{Forks} & \textbf{Last Commit} & \textbf{Issues} \\
        \hline
            Pysyft \cite{fl_framework:pysyft} & 0.9.0 & two weeks ago & 9.4k & 2k & Same Day & 2
        \\
        \hline
            Tensorflow Federated \cite{fl_framework:tensorflow_federated} & 0.85.0 & two days ago & 2.3k & 578 & Same Day & 168
        \\
        \hline
            FedML \cite{fl_framework:fedml} & 0.8.9 & 11 months ago & 4.1k & 776 & 3 months ago & 118
        \\
        \hline
            Flower \cite{fl_framework:flower} & 1.10.0 & 3 weeks ago & 4.7k & 815 & Same Day & 284
        \\
        \hline
            OpenFL \cite{fl_framework:openfl} & 9.3.4 & last month & 1.9k & 426 & Same Day & 256
        \\
        \hline
    \end{tabular}
    \captionof{table}{Updated FL Framework Comparison} 
    \label{table:updated_fl_framework_comparison}
\end{changemargin}


Table \ref{table:updated_fl_framework_comparison} shows our updated FL Framework comparison.
Note that we took these stats on 16.08.2024.
These FL frameworks are in active development. 
Only FedML has not been updated for several months now.

Saidani's main original contribution was a novel FL benchmarking suite called FMLB (Federated Machine Learning Benchmark).
He developed it to evaluate and compare the mentioned FL frameworks efficiently.
His previous analysis and summary of existing frameworks were sound and helpful.
However, we are critical of his evaluation results, especially the poor performance of Flower surprised us.
We tried to replicate his experiments, but his provided code \cite{tum_fl_framework_thesis_github}
lacks instructions on how to set up this benchmark application.

We simulated the experiments with the latest official flower version of that time,
and made sure to stick as close as possible to the same experimental setup and configuration.
Our findings show very different results.
Flower manages to solve the experiment quickly and efficiently.
Our results match the verdicts of other works comparing FL frameworks, such as \cite{comparative_analysis_of_fl_frameworks} or \cite{comprehensive_study_fl_frameworks_degree_project}.
\cite{comparative_analysis_of_fl_frameworks} is the latest work that compares FL frameworks that we considered,
and its verdict is that Flower even outperforms all its competition.

We decided to use Flower as the FL framework for FLOps.
% ~1 page explanation what Flower is - how it works - and why we choose to use it
% important - mention cons of flower - and limitations (e.g. lack of HFL, MLOps tooling, containerization/orchestration - mention github example where they tell the user to create a docker image from scratch)
\subsection{Flower}



% manage all these moving FL components - need orchestration
% FL primarily focused on edge - use edge friendly orchestrator
% -> Oakestra


% FL has many components -> need to manage those
% Motivate Orchestration -> MLOps/DevOps - key thing is automation -> Orchestration
\section{Orchestration}

% "basics of orchestrations" - i.e. form VM to containers/docker -> towards Orchestrators like K8S
\subsection{Fundamentals}

% show that it makes sense to containerize and orchestrate ML
% https://www.notion.so/oakestra-team/Studying-the-Practices-of-Deploying-Machine-Learning-Projects-on-Docker-9852581582c54e2a837d820752059830?pvs=4
% papershttps://www.notion.so/oakestra-team/Performance-Evaluation-of-Deep-Learning-Tools-in-Docker-Containers-198fb74e3651480a913eb48b4c59cfb5?pvs=4
% https://www.notion.so/oakestra-team/ECO-Harmonizing-Edge-and-Cloud-with-ML-DL-Orchestration-dc49ce91255541ec925e48ae1f857052?pvs=4
\subsection{ML Containerization \& Orchestration}



% FL is primarily on edge -> use Orchestrator that is specialized for Edge -> Oakestra
% motivate why oakestra was chosen + explain it briefly 
\subsection{Oakestra}


% The 2-3 papers that are "very" simliar to FLOps
% the large suite of FL papers should be covered in background chapter before
\section{Related Work}

Only two previous works \cite{paper:fl_toward_on_demand_client_deployment_at_edge, paper:global_fl_platform_for_iot} of those mentioned in \ref{subsection:fl_research} resemble FLOps.
Both also noticed the lack in research regarding deploying ML and FL capabilities dynamically via containers.
They use different technologies and offer different functionality than FLOps.
They focus on other aspects and do not incorporate MLOps tools, automatic image builds, or automatic deployment of trained model inference servers.

\subsection{On the feasibility of Federated Learning towards on-demand client deployment at the edge}
In 2023, Chahoud et al. \cite{paper:fl_toward_on_demand_client_deployment_at_edge} proposed a three-layered FL architecture running on Kubernetes.
Each following component has a matching image in DockerHub.
Newly joining devices can simply pull these images.
\vspace{5mm}
\newline
\textbf{Server}\newline
The first layer is the server or service provider.
The server has various managerial responsibilities.
It serves container images to voluntary devices and maintains secure connections to other layers.
The server is the aggregator that manages the global model.
Together with the mini-servers it determines which nodes should form a cluster.
It handles service deployments and client selection after receiving requests from mini-servers.

Various components are part of the server.
An oracle engine supports building the base model that will be send out to clients.
The oracle determines what type of ML technique is required based on the environment, such as classic ML or DL.
An enhanced FL aggregator handles stragglers and missing updates from failed learners.
The aggregator uses a threshold to determine if an update should be included or discarded.
A Kubeadm environment initializer that turns devices into mini-servers.
This decision is based on available devices and the level of client mobility.
From there followup setup steps are performed, such as cluster creation and population or container deployment on worker nodes.
A communication manager upholds a stable connection between the different layers.
A orchestrator manager administers the second layer mini-servers.
It can dynamically determine and change which devices should be mini-servers.
This task helps to gather better data for FL.
\vspace{5mm}
\newline
\textbf{Orchestrators / mini-servers}\newline
Mini-servers handle Kubeadm clusters and surveil device movements.
They deploy containers and add workers to clusters.
Similar to cluster-orchestrators in Oakestra, mini-servers distribute management tasks and workloads among each other and away from the server.
A mini-server containers a client profiler component and a client manager.
The former gathers worker metrics, and the latter informs the server about changes in the environment.
Relevant changes include joining and leaving clients.
The client manager also protects against client starvation.
\vspace{5mm}
\newline
\textbf{User Devices}\newline
User devices are the FL learners.
They contain a client profiler that shares the machine metrics with a corresponding mini-server.
The profiler also informs the mini-server in case of failures.
\vspace{5mm}
\newline
Open challenges and future work include more efficient and secure selection algorithms.
More sophisticated logic is required to select learners for training to optimize FL results via data heterogeneity.
Selecting devices to become mini-servers is a potential security hazard.
Currently the authors assume mini-servers to be trustworthy and reliable.

Their evaluation results show that their FL solution is only 10\% worse than the centralized alternative.

The similarities between this work and FLOps are the following.
Both focus on making FL easy to use and do not focus on optimizing models or algorithms.
They enable on-the-fly dynamic deployment and setup of FL components on unprepared devices via containerization technologies.
Both provide prepared container images via public registries.
This work's mini-servers and "root" server resemble Oakestra's root and cluster orchestrators.
Oakestra is a dedicated orchestrator while this work's components are auxiliaries with less features.

This work used a different orchestrator, FL framework (augmented FedML) and image registry.
FLOps supports classic and HFL.
This work only supports HFL.
They used a single hardcoded dataset and ML model for evaluation.
This work does not offer different scenarios or utilize dedicated MLOps features and techniques.
FLOps allows users to build and train various custom ML code.
This works places its focus on 6G and actual real world movement of people, whereas FLOps is more general and feature rich.


\subsection{Towards Developing a Global Federated Learning Platform for IoT}
In 2022, Safri et al. \cite{paper:global_fl_platform_for_iot} developed a prototype to improve FL on IoT devices.
This work enables distributed ML model deployment, federated task orchestration, and monitoring of system state and model performance.
They called their approach FedIoT.
Their three-layered architecture resembles the one from \cite{paper:fl_toward_on_demand_client_deployment_at_edge} and Oakestra.
Their root server/orchestrator is called global orchestrator.
It acts as an FL aggregator and dynamically configures and deploys local orchestrators via an API.
Their local orchestrator is not equivalent to cluster orchestrators or mini-servers.
This work focuses on enterprise IoT.
IoT devices are usually not capable of handling common ML training due to their limited resources.
This work acknowledges this and uses the IoT devices only as data providers but not as learners.
Therefore, this work performs classic FL instead of HFL.
Local orchestrators are learners in this architecture.
They need to be in the proximity of IoT devices to be able to receive their data.
Additionally, they provide customizable data preprosessing and evaluation code to be injected via the API.

This work offers additional tooling, such as a custom compressor and monitoring.
The compressor is a dedicated component to reduce the size of large files.
Monitoring agents are deployed on the local and global orchestrators that measure resources and CO2.
A custom GUI presents these metrics.

As future work the authors wanted to add more FL algorithms and add more sophisticated logic to select participants for training based on the monitored metrics.

There are obvious similarities between this work and FLOps/Oakestra.
Both want to provide a one-in-all solution to perform FL on tangible devices via containerization and orchestration.
They want to automate setup, dependency management, configuration, and metric gathering.
Additionally, they want to improve comprehension and observability by providing a GUI.

This work differs compared to FLOps in multiple ways, besides FLOps larger set of features.
As already mentioned above, this work only offers classic FL and has a different and less mature architecture than FLOps thanks to Oakestra.
This work is very short, thus lacks details and readability.
It has no open sourced code to inspect and replicate its implementation.
FLOps has this thesis documenting it in great detail and is fully open source.
This works tries to implement all components by itself from group up, such as orchestration, monitoring, and FL.
FLOps utilizes and combines existing sophisticated solutions to offer higher quality features and performance.
For example, this work's GUI is a simple Grafana dashboard, that offers a lot fewer features and is read-only.
FLOps utilizes MLflow to provide a sophisticated suite of MLOps tools and functionalities.
