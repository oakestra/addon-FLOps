\section{Motivation}

Building or contributing to a novel FL framework or library focusing on 
the previously mentioned challenges areas could soften or entirely alleviate those problems. 
We are talking about a tool that sees Docker and Kubernetes as role models
and strives to be comparable to them but for the discipline of FL.
It should specialize in the setup, deployment, component management, and automation,
in short, FL orchestration.
Allowing researchers, developers, and end-users to set up, perform, reproduce,
and experiment with FL in a more accessible way.

The goal of this tool should be to automate and simplify complex tasks,
reducing the required level of expertise in various domains, ranging from ML/FL,
dependency management, containerization technologies, and automation to orchestration.
Such a tool would empower less experienced individuals to participate and contribute to the field of FL.
As a result, FL could be adapted and used by more people in more areas.

This tool would streamline and accelerate existing workflows and future progress
by utilizing reliable automation to avoid error-prone manual tasks.
With its potential to optimize, standardize and unify workflows,
our envisioned tool could become a significant part of the emerging FL ecosystem,
contributing to the development and progress of the entire field.