\section{Objectives}

This motivation allows us to distill the following key objectives for such a tool.
\vspace{5mm}
\newline
\textbf{Improve Accessibility}\newline
Making FL more accessible by abstracting away and automating complexities,
enables a larger number of individuals to engage with it.
Helping FL to expand to more areas will increase its usage and user base,
thus raise the general interest and relevance for its field which should aid its development.
\vspace{5mm}
\newline
\textbf{Add Automation}\newline
Automating tedious, error-prone, and repetitive manual tasks that are necessary
to perform FL will free up time and resources for more critical work that leads
to further advancements.
\vspace{5mm}
\newline
\textbf{Prioritize tangible Applicability}\newline
As we discuss in (\ref{subsection:fl_research}) FL is struggling from a gab
between research/virtual-simulation and practical application in real production environments.
This tool should focus on being usable in real physical conditions on distributed devices.
It should be feasible to incorporate this tool into existing workflows.
\vspace{5mm}
\newline
\textbf{Embrace Plasticity}\newline
Because FL is such a young field it faces constant change.
Naturally, our tool should welcome change in the form of extendability
and adaptability.
This tool should be flexible and be applicable in a myriad of use cases and scenarios.
It should be easy to modify to accommodate evolving needs.