% HFL on monolith
% non-HFL (base-case) on multi-cluster
% HFL on multi-cluster
% diff HFL configurations & discussion

% DONT forget to inlcude the proof of concept here first!
\subsection{Multi-cluster \& HFL}

\subsubsection{Classic FL on the Multi-Cluster Setup}

Before analyzing how real HFL perform on multiple clusters we need to verify that the second setup is capable of classic FL and compare it to the monolith setup.
Observing the aggregated sum of metrics from all three devices leads to flat and less insightful plots.
For this reason many following graphs will depict metrics by device.
Figure \ref{fig:eval_7_cpu} shows the CPU utilization of the cluster setup during experiment (7).
It depicts how the control plane root device only manages components but does not perform any heavy operations as intended.
Its CPU utilization is constantly at a very low level.
The orchestrator selected and deployed the image builder service on cluster-B.
That is why it is the only busy device during the initial image build phase.
Notably, this scheduling decision seems to be deterministic.
Cluster-B was selected every single run by the orchestrator.
FLOps does not tell the orchestrator to use the same cluster for the image builders.
Once FL actor deployment and training starts the CPU utilization is distributed among both cluster nodes.
Figure \ref{fig:eval_7_mem} shows the memory utilization of the devices.
The memory stays stable when no workloads are performed on a device, which is the case for the root at all times and for cluster-A while cluster-B is building the images. 

Figure \ref{fig:eval_7_disk_space} shows the disk space increase of the devices.
It shows how cluster-B's disk space is increasing during the build process due to the dependencies and layers that are pulled and build.
As well as a drop after the build process is finalized and the builder service undeployed.
In the middle of the build process cluster-B pushes the built base image to the root that hosts the image registry.
There is no significant increase when cluster-B pushes the FL-actor images because of the reuse of the common base-layers.
The disk space of cluster-A only starts increasing when the FL-actors are deployed on it.
Matching the disk space changes Figures \ref{fig:eval_7_net_received} and \ref{fig:eval_7_net_send} show the received and send network changes on the devices.

The FLOps stages of experiment (7) do not show any remarkable outliers and resemble the base-case.
The only difference is that due to the weaker hardware the project and all its stages take longer.
The final training results are equivalent to (1).

\begin{figure}[H]
    \begin{adjustwidth}{-0.2\paperwidth}{-0.2\paperwidth}
        \centering
        \includegraphics[width=0.85\paperwidth]{evaluations/multicluster_hfl/eval_9_7_simple_multicluster_cpu_split.png}
        \caption{Experiment 7: CPU Utilization}
        \label{fig:eval_7_cpu}
    \end{adjustwidth}
\end{figure}

\begin{figure}[H]
    \begin{adjustwidth}{-0.2\paperwidth}{-0.2\paperwidth}
        \centering
        \includegraphics[width=0.85\paperwidth]{evaluations/multicluster_hfl/eval_9_7_simple_multicluster_mem_split.png}
        \caption{Experiment 7: Memory Utilization}
        \label{fig:eval_7_mem}
    \end{adjustwidth}
\end{figure}

\begin{figure}[H]
    \begin{adjustwidth}{-0.2\paperwidth}{-0.2\paperwidth}
        \centering
        \includegraphics[width=0.85\paperwidth]{evaluations/multicluster_hfl/eval_9_7_simple_multicluster_disk_split.png}
        \caption{Experiment 7: Disk Space}
        \label{fig:eval_7_disk_space}
    \end{adjustwidth}
\end{figure}

\begin{figure}[H]
    \begin{adjustwidth}{-0.2\paperwidth}{-0.2\paperwidth}
        \centering
        \includegraphics[width=0.85\paperwidth]{evaluations/multicluster_hfl/eval_9_7_simple_multicluster_net_received.png}
        \caption{Experiment 7: Received Network}
        \label{fig:eval_7_net_received}
    \end{adjustwidth}
\end{figure}

\begin{figure}[H]
    \begin{adjustwidth}{-0.2\paperwidth}{-0.2\paperwidth}
        \centering
        \includegraphics[width=0.85\paperwidth]{evaluations/multicluster_hfl/eval_9_7_simple_multicluster_net_send.png}
        \caption{Experiment 7: Send Network}
        \label{fig:eval_7_net_send}
    \end{adjustwidth}
\end{figure}

\subsubsection{Minimal HFL base-case on a monolith Cluster}