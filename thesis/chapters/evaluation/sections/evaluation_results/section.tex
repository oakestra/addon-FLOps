\section{Results}

We split our findings into the following subsections.
The basics explain and showcase the simplest base-case and what graphs we use to visualize our findings for all experiments.
Afterward, we analyze how different variables change the image build processes.
Penultimately, we discuss how different ML repositories, frameworks and datasets perform in FLOps.
Lastly, we focus on HFL and verify that our custom novel solution is sound.

\subsection{Basics}

This subsection introduces the different plots we use to visualize all our results.
It is more verbose and includes explanations and additional plots than the following subsections, which are similar.
This part focuses on experiment (1).
All mentioned graphs are also available for all other experiments.
Showcasing all of them would heavily bloat this work.
Thus, we omit them.
These graphs and the underlying CSV files are available in the extended OAK CLI code evaluation folder \cite{cli_code}.

\subsubsection{CPU \& Memory}

\begin{figure}[h]
    \begin{adjustwidth}{-0.2\paperwidth}{-0.2\paperwidth}
        \centering
        \includegraphics[width=0.99\paperwidth]{eval_1_simplest_cpu_mem.png}
        \caption{Experiment 1: CPU \& Memory Utilization}
        \label{fig:eval_1_simplest_cpu_mem}
    \end{adjustwidth}
\end{figure}

Graph \ref{fig:eval_1_simplest_cpu_mem} shows the recorded CPU and memory utilization across a project's lifetime with stage information.
This information shows the mean of all ten evaluation runs with a 95\% confidence interval.
The colored areas represent specific project stages.
The graph unveils that the memory utilization stays relatively stable throughout a project and only slightly increases during FL training and the non-base image FL actor builds.
Note that the FL-Actors Image Build stage represents the entire image build and push process, which includes the base image and the actor images.
Deployment stages represent time frames in which components and services for the next stage are created and deployed via the FLOps manager and orchestrator, but these services/images do not yet start their workloads.
Most stages do not utilize much of the available CPU except during the FL actor deployment stage and FL training, which makes sense because this experiment uses the CPU for training.
On average, this simple base case takes 12 minutes to complete on the monolith system.

\begin{figure}[h]
    \begin{adjustwidth}{-0.2\paperwidth}{-0.2\paperwidth}
        \centering
        \includegraphics[width=0.99\paperwidth]{eval_1_simplest_cpu_boxviolin.png}
        \caption{Experiment 1: CPU Utilization by Stage}
        \label{fig:eval_1_simplest_cpu_boxviolin}
    \end{adjustwidth}
\end{figure}

Figure \ref{fig:eval_1_simplest_cpu_boxviolin} shows a box-violin plot of the CPU utilization for different experiment stages.
The largest median CPU consumption occurs in the FL training stage.
What is remarkable is that the deployment stage for the FL actors (Aggregator Deployment in the plot) also has high CPU utilization.
Multiple services' rapid creation, deployment, and orchestration can explain this.
Both image build stages have many outliers, indicating that the build process is highly heterogeneous.

\begin{figure}[h]
    \begin{adjustwidth}{-0.2\paperwidth}{-0.2\paperwidth}
        \centering
        \includegraphics[width=0.95\paperwidth]{eval_1_simplest_memory_boxviolin.png}
        \caption{Experiment 1: Memory Utilization by Stage}
        \label{fig:eval_1_simplest_memory_boxviolin}
    \end{adjustwidth}
\end{figure}

Figure \ref{fig:eval_1_simplest_memory_boxviolin} is similar to the previous plot but depicts the memory utilization per stage.
The FL training stage is the most consuming.
All other stages are below 60\% memory utilization except for the FL actors builder and its deployment stages, which have multiple outliers that reach the high 70s.
Unlike CPU outliers, which only lead to throttling, memory outliers can lead to out-of-memory exceptions and failures.
Thus, it is vital to be aware of such behavior.

\pagebreak
\subsubsection{Normalization}

Individual evaluations range in duration.
The reasons for this can be manifold, such as different current image pull speeds due to local network conditions or remote registry loads.
\begin{figure}[h]
    \begin{adjustwidth}{-0.2\paperwidth}{-0.2\paperwidth}
        \centering
        \includegraphics[width=0.99\paperwidth]{eval_1_simplest_explanation_shift.png}
        \caption{Individual Experiment Run Durations}
        \label{fig:eval_1_simplest_explanation_shift}
    \end{adjustwidth}
\end{figure}
Figure \ref{fig:eval_1_simplest_explanation_shift} shows the memory usage of individual evaluation rounds over time.
It is clear to see that the usage pattern is the same but shifted over time.
We normalized the average of these rounds to compare and visualize them properly.
Otherwise, stage borders become duplicated and overlapped, means and confidence intervals do not lead to meaningful outcomes, and the graphs are more confusing than helpful.

\pagebreak
\subsubsection{Disk Space}

\begin{figure}[h]
    \begin{adjustwidth}{-0.2\paperwidth}{-0.2\paperwidth}
        \centering
        \includegraphics[width=0.99\paperwidth]{eval_1_simplest_disk_pace_linegraph.png}
        \caption{Experiment 1: Disk Space Changes over Time}
        \label{fig:eval_1_simplest_disk_space}
    \end{adjustwidth}
\end{figure}

Graph \ref{fig:eval_1_simplest_disk_space} shows how the disk space changes over the project's lifetime.
There was a total average increase of 14 GB.
It starts with the Image-Builder-Deployment stage, where the Image-Builder image is pulled.
Many components and dependencies are downloaded and pushed to the registry on the same monolithic device during the FL actors' build process.
The jump in disk space in the aggregator (FL actors) deployment stage is because containerd needs to pull these build images.
Thus, the monolith will have the same image in the image registry and its local containerd image context.
During FL training, the disk space remains the same, which verifies that even when using FLOps with demanding lengthy training configurations and models, no disk space issues will arise due to its training process.
Disk space occasionally goes down due to the system's garbage collection, which is independent of FLOps or Oakestra.

\begin{figure}[h]
    \begin{adjustwidth}{-0.2\paperwidth}{-0.2\paperwidth}
        \centering
        \includegraphics[width=0.95\paperwidth]{eval_1_simplest_disk_stages_box.png}
        \caption{Experiment 1: Disk Space Changes per Stage}
        \label{fig:eval_1_simplest_disk_space_stages}
    \end{adjustwidth}
\end{figure}
Figure \ref{fig:eval_1_simplest_disk_space_stages} shows the disk space changes per stage.
The aggregator (FL-actors) deployment stage takes up the most space.
The trained-model image deployment stage is minimal compared to the first builder deployment because of the local containerd image storage.
Containerd pulls the builder image once and reuses it afterward.
Especially during image-building processes, much space is freed up again due to garbage collection.

\subsubsection{Network IO}

\begin{figure}[h]
    \begin{adjustwidth}{-0.2\paperwidth}{-0.2\paperwidth}
        \centering
        \includegraphics[width=0.99\paperwidth]{eval_1_simplest_network_linegraph.png}
        \caption{Experiment 1: Net-IO over Time}
        \label{fig:eval_1_simplest_net_io}
    \end{adjustwidth}
\end{figure}

Figure \ref{fig:eval_1_simplest_net_io} shows the network loads received and sent over a project runtime.
The most significant increases are during built image pushes.
They occur around minute five when the base image is pushed, at the end of the FL-Actors Image Build stage phase, and when the trained-model image build stage bleeds into its deployment stage.
These values should be strictly increasing due to the accumulative nature of network IO counters.
This plot shows dips.
The reason for this is connected to removed containers.
The displayed lines are the sums of all received and sent traffic detectible on all network interfaces.
This includes virtual network interfaces of containers.
For example, there is a noticeable decrease between the FL-Actors image build and aggregator deployment stages.
I.e., between the moment the builder service finishes pushing its images and terminates and before these build services get deployed.
The image builder container had its own virtual network interface, which was included in the total sum as one of the lines.
Once the container is removed, its virtual network interface is also deleted, and its accumulated net IO will be removed from the sum of the following system metrics scrapes.
We omit to present violin-box plots detailing net-io because this additional information does not lead to any significant insights.

\subsubsection{Stage Runtimes \& Training Results}

\begin{figure}[h]
    \begin{adjustwidth}{-0.2\paperwidth}{-0.2\paperwidth}
        \centering
        \includegraphics[width=0.99\paperwidth]{eval_1_simplest_stage_durations.png}
        \caption{Experiment 1: Stage Durations}
        \label{fig:eval_1_simplest_stage_durations}
    \end{adjustwidth}
\end{figure}

Graph \ref{fig:eval_1_simplest_stage_durations} shows each stage's average duration.
The FL training stage is relatively short because the training configuration is minimal.
The image build stages both take up the vast majority of time.
The FL-actors image build process involves more images with complex dependency resolutions.
Thus, this build stage takes over twice as long as the trained-model image build.

Figures \ref{fig:eval_1_simplest_accuracies} and \ref{fig:eval_1_simplest_loss} show the accuracies and losses of the trained models after each evaluation round.
They prove that FLOps can train ML models in a stable way.

\begin{figure}[H]
    \centering
    \includegraphics[width=0.99\textwidth]{eval_1_simplest_accuracy.png}
    \caption{Experiment 1: Trained Model Accuracies}
    \label{fig:eval_1_simplest_accuracies}
\end{figure}

\begin{figure}[H]
    \centering
    \includegraphics[width=0.99\textwidth]{eval_1_simplest_loss.png}
    \caption{Experiment 1: Trained Model Loss}
    \label{fig:eval_1_simplest_loss}
\end{figure}



% image build related 
% - speeds 
% - sizes 
% - cache 
% - baseimages 
% - multiplatform
\subsection{Image Builder}
TODO

% depens how much progress I can make here besides already established pytorch
\subsection{Different ML Frameworks/Libraries \& Datasets}

% HFL on monolith
% non-HFL (base-case) on multi-cluster
% HFL on multi-cluster
% diff HFL configurations & discussion

% DONT forget to inlcude the proof of concept here first!
\subsection{Multi-cluster \& HFL}