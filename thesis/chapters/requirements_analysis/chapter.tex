\chapter{Requirements Analysis}

This chapter performs requirements engineering.
It uses the problem statement \ref{section:problem_statement}, objectives \ref{section:objectives}, and weaknesses found in the FL field \ref{subsection:fl_research} as input to elicit requirements.
These requirements are specified, analyzed, and concretized. 
This chapter starts by analyzing and deriving functional and nonfunctional requirements.
It utilizes (UML) models and use cases to explain the proposed system's functionalities and structure.
Its goal is to explain the new system's workflows, processes, and structural relationships without delving into concrete ways of realizing these goals.
This chapter focuses on the application domain, which represents the proposed system and its surrounding environment.
It is crucial to understand the basic behavior, reasoning, and environment of the system before working out how to realize these goals in a concrete manner.
The contributions section \ref{section:contributions} provides an overview of FLOps's functionalities and features and how they realize the discussed objectives.
This chapter follows the Requirements Analysis Document Template by Brügge et al. \cite{book:bruegge}.

\section{Proposed System}

\input{chapters/requirements_analysis/subsections/functional_requirements.tex}

\input{chapters/requirements_analysis/subsections/nonfunctional_requirements.tex}

\section{System Models}

% TODO paraphraze!
In this section, we will analyze the formulated requirements. We start with user stories, i.e. scenarios that describe the new system which implements the requested changes.
We continue by illustrating the structure and workflows of the new system by creating various UML models.
Finally, we take a look at the GUI that shows how the user will interact with the new system.

\subsection{Scenarios}

% TODO paraphraze!
The goal of scenarios is to enhance the understanding of the proposed system by looking at a concrete set of common use cases [BD09].
These scenarios are also written in natural language to further increase the comprehension of the planed end user experience while working with the new system.
Scenarios can also be used to build models. We differentiate between two kinds of scenarios.
Demo scenarios that illustrate the achieved use cases of the new system.
Visionary scenarios portrait an almost utopian experience that will not be realized by this thesis, but lay the foundation for future work.

\subsubsection{Visionary Scenario}
TODO
% Maybe split scenarios up by target group - enduser no experience - FL researcher - FL dev
\subsubsection{Demo Scenario A}

\subsubsection{Demo Scenario B}

\subsection{Use Case Model}
% TODO paraphraze!
UML Use Case diagrams are used to visualize the available system functions from the user perspective, i.e. they illustrate the previously described functional requirements [BD09].
They illustrate all use cases instead of a subset like the previously covered scenarios.

\subsection{Analysis Object Model}
\subsection{Dynamic Model}
