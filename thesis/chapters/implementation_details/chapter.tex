\chapter{Implementation Details}

After discussing requirements and simplified system models this chapter discloses significant implementation details that power FLOps.
FLOps is mostly implemented in Python.
Our goal is to use state-of-the-art tools for FLOps' implementation.
We analyzed and compared different open-source libraries and tools.

\begin{figure}[h]
    \begin{adjustwidth}{-0.175\paperwidth}{-0.175\paperwidth}
        \centering
        \includegraphics[width=0.975\paperwidth]{python_libraries.png}
        \caption{Screenshot of internal python projects/libraries analysis}
        \label{fig:python_projects_libraries}
    \end{adjustwidth}
\end{figure}

Figure \ref{fig:python_projects_libraries} shows a screenshot of a subset of our internal comparison.
We restrain from exploring and discussing our findings verbatim to avoid bloating the thesis.
Especially because this landscape is constantly changing and better options might exist since our analysis. 
We carefully considered every dependency that FLOps relies upon.

\section{User Interactions with the FLOps Manager}
This section details how users can interact with the FLOps manager.
It shows the currently available API endpoints and the SLA structure.

    \subsection{API}
    The FLOps manager API is implemented via the Flask-OpenAPI3 framework.
    This is a rather young and small framework that aligns Flask with OpenAPI3.
    It uses Pydantic to verify data and automatically generate REST API and OpenAPI documentation that is compatible with popular frameworks such as Swagger.
    We use waitress a production-quality WSGI server to host the manager's API.

    \subsubsection{Currently Available Endpoints}
        \textbf{/api/flops/projects}\newline
        This POST endpoint triggers a new FLOps project.
        It expects users to provide a SLA json with the required project configurations and a bearer token authorizing the user on the orchestrator.
        If no matching images exist a image builder is created and deployed.
        If an adequate image already exists the request concludes straight away.
        The user receives a confirmation that the new project has successfully started.
        \vspace{5mm}
        \newline
        \textbf{/api/flops/tracking}\newline
        The tracking endpoint allows users to spawn their individual tracking servers at will independently from an active project.
        Usually, a tracking server is created during FL training.
        This GET endpoint returns the tracking server / GUI URL.
        \vspace{5mm}
        \newline
        \textbf{/api/flops/database}\newline
        This DELETE endpoint only allows admins to reset the FLOps database.
        Otherwise the entire FLOps management suite needs to be reset.
        It returns a confirmation for the user.
        \vspace{5mm}
        \newline
        \textbf{/api/flops/mocks}\newline
        This POST endpoint creates mock data providers and deploys them on fitting learner machines.
        These date providers are discussed later on in this thesis.
        Similarly to the project this endpoint returns a confirmation to the user.
        \vspace{5mm}
        \newline

    \subsection{SLAs}


\section{Image Building}

    \subsection{Dependency Management}
    % Python dependency explanation

    \subsection{ML Frameworks and their Sizes}
    % talk about the support of different frameworks
    % very brief overview how many there are
    % mention how large these dependencies and images are

    \subsection{Image Builders}
    % Need for Buildah vs Docker

    \subsection{Multi Platform}

    \subsection{FLOps Image Builder Details}
    % How does FLOps builds its image concretely
    % base images
    % also mention how trained images are build - mlflow + dockerfiles + buildah

\section{Local Data Management}
% explain in detail how this works
% mention mock data providers
% mention apache arrow, parquet, flight
% reuse/augment existing graphics

\section{MLOps via MLflow}

    \subsection{MLOps Components \& Architecture}
        % Explain how the MLOps architecture works in FLOps
        % Tracking Server as Proxy for storages, etc.
        % Explain how and where the results are stored

    \subsection{GUI}
        % Showcase how the GUI looks like and how it can be used

\section{Clustered HFL}
    % Explain in detail how this novel approach works

\section{CLI}
    % brief overview, etc - maybe even a use case model, etc.
    % screenshots

