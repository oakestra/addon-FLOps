% \section{Limitations \& Future Work}
\section{Current Status, Limitations \& Future Work}
% Build-times (mention experimental new build tools (slim, nydus, etc.))
% Dependency Hell (Build/MLflow,etc)
% Occasional Flakiness due to implementation (new to pydantic)

% https://aneescraftsmanship.com/circle-symbols%E2%97%8B%E2%97%8F%E2%97%8D%E2%97%97%E2%97%94%E2%97%99%E2%A6%BF-in-latex/
\begin{itemize}
    \item [\faCircleO] \textbf{Not Implemented}: FLOps does not meet the requirement in its current form. 
	\item [\faDotCircleO] \textbf{Partially Implemented}: FLOps only partially meets the requirement.
	\item [\faArrowCircleRight] \textbf{Implemented MVP}: FLOps fulfils the requirement in a minimal viable way. Features can be further extended.
	\item [\faCircle] \textbf{Fully Implemented}: FLOps fully realizes the requirement.
\end{itemize}

\input{tables/conclusion_functional_reqs_status.tex}

Table \ref{table:status_functional_reqs} depicts how well the current FLOps implementation fulfills its functional requirements.
We count FR-1.3 and FR-2 as MVP implemented because FLOps does support different FL scenarios such as classic and clustered HFL as well as various project configurations.
This support is not exhaustive.
There exist multiple additional FL strategies, algorithms, and Flower configurations that can and should be analyzed and added to future FLOps versions to allow a wider range of FL capabilities.
FLOps currently supports Scikit-learn, Pytorch, and Tensorflow.
Other ML flavors can be easily added by extending the underlying minimal enum structure.
We regard FR-6's inference serving as MVP implemented because enabling inference serving is not a primary focus point of FLOps.
Instead inference serving is FLOps currently last post-training step thus we did not invest too much time into this feature.
To be seen as fully implemented it should be thoroughly tested and investigated.
MLflow offers different ways of turing trained models into inference servers.
Analyzing and comparing these variations and letting users decide how to create inference servers is a valid piece of future work.

\input{tables/conclusion_nonfunctional_reqs_status.tex}

Table \ref{table:status_nonfunctional_reqs} shows the fulfillment grade of nonfunctional requirements by the current FLOps version.
FLOps only partially realizes NFR-2.2.2.
During the development of FLOps we had to introduce several minor changes and additions to Oakestra for it to handle FLOps workflows.
FLOps project structure follows the application and service structure of Oakestra which seems to be different to other orchestators like Kubernetes.
We did not try to run FLOps via another orchestrator like Kubernetes due to the restricted time.
Therefore, we cannot tell how straightforward it is to use FLOps with other orchestrators.
Nontheless, FLOps communicates via strict decoupled APIs and SLAs with Oakestra, thus replacing or extending these interaction points to other orchestrators should be straightforward.
In addition, Jabok Kempter's work (TODO) to combine Oakestra and Kubernets alludes that FLOps can be run on Kubernetes even as a Oakestra addon.
We marked NFR-3.1 as MVP implemented and NFR-3.2 as partially implemented because FLOps is capable of running on monolith and small multi-cluster setups with varing numbers of FL-actors.
We did not try to run FLOps on a truly large scale, such as several hundred or thousand devices yet, thus cannot indefinitely confirm that FLOps is capable of handling large scale deployments.
Regarding availability, FLOps includes several mechanism of communication and error/event handling.
However, if FLOps services or applications are tinkered with or removed/deleted from an external non FLOps source current FLOps is incapable of reacting accordingly.
We hope to change this once FLOps is connected to Mahmoud ElKodary's Oakestra addon marketplace and hooks so it is able to responde and receive these vital events and react accordingly.

\subsubsection{Security \& Privacy}
Security and privacy are foundational concerns of FL.
Our priority with FLOps was to create and verify its foundational architecture and components.
Due to the lack of access to proper certificates and to accelerate FLOps' development we omitted privacy and security concerns.
This includes the use of HTTP instead of HTTPS or the lack of supported FL security features such as secure aggregation.
This aspect is one of the most important for future work.
The good news are that Flower, the image registry, and local data management Apache suite all come with security features that simply need to be configured properly.

% Security & Privacy 
% GPU
% Personalized FL <- mention papers here (make an introduction here)
% Extended/Modified HFL
% More FL - diff directions - diff algos
% More ML Frameworks & Datasets Experiments
% Experimentation with edge focused ML frameworks, PI, etc (multi-platform/edge)
\subsubsection{Federated Learning via FLOps}

% More automation via more CLI integration, etc
% Integrate with Mahmouds work
% Evaluated primarily on Monolith machine - the resources could be analyzed in greater detail (per container/service) remove host,etc traffic/consumption
\subsubsection{Complementary Components \& Integrations}